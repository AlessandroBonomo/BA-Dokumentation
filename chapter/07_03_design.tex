\subsection{Design}
\subsubsection{Architektur und Abhängigkeiten}
Bei der Wahl von Technologien, Tools und Frameworks achtete man besonders darauf, 
%TODO
\subsubsection*{Applikations Architektur}
Für die Umsetzung der Anwendung hatte man zwei Optionen. Entweder man entwickelt eine Native App, also eine klassische Client-Server Anwendung, oder eine Web Anwendung. \\

Native Apps haben zwar einige Vorteile. Die Integration mit anderen Anwendungen wäre so zum Beispiel einfacher. Zudem wäre die Monetarisierung 	auch um einiges einfacher, denn eine Native App könnte man sehr gut im App Store platzieren. So würde auch gleich die Bekanntheit der Applikation steigen. \\

Die Entscheidung bei der Entwicklung von Aufgaben-Coaching viel jedoch sehr schnell auf eine Web Anwendung. Viele Schulen haben zwar Apple iPads im Einsatz. Trotzdem kann es Schulen geben, welche mit Android Devices arbeiten. Da die Anwendung aber über den Browser läuft, muss nur eine Anwendung entwickelt werden. Bei einer Native App müsste man für Apple und Android je eine Anwendung entwickeln. Für grosse Firmen ist dies kein Problem, aber möchte man die Idee als Startup weiter verfolgen, hätte man weder die Zeit noch das Kapital um gleich zwei Apps zu entwickeln. \\
Zudem kann man mit Web Anwendungen die Supportkosten relativ tief halten, was auf Schulen vielleicht noch ein bisschen attraktiver wirkt. Die Anwendung kann direkt im Browser aufgerufen werden und es muss nichts im vorhinein auf dem Device installiert werden.

% http://www.differencebetween.net/technology/software-technology/difference-between-client-server-application-and-web-application/	
% https://www.adjust.com/blog/native-app-vs-progressive-web-app/

\subsubsection*{Programmiersprache}
Für das Web Development gibt es mehrere Programmiersprachen, die zur Auswahl standen. Die Wohl bekanntesten darunter sind:

%https://en.wikipedia.org/wiki/Web_framework

\begin{itemize}
	\item Python (Django / Flask)
	\item C\# (ASP.NET Core)
	\item Java (Spring)
	\item JavaScript (Express.js / Node.js / Sails.js)
	\item PHP (CakePHP / CodeIgniter / Laravel)
	\item Perl (Catalyst / Mojolicious)
	\item Ruby (Ruby on Rails)
\end{itemize}

%TODO Satzstellung anschauen

Die Wahl fiel schnell auf Python, da wir beide am meisten Knowhow über Python verfügen. \\
Gemäss Developer Survey Results von Stack Overflow ist Python (41.7\%) etwas bekannter als Java (41.1\%). Nur JavaScript (67.8\%) ist noch bekannter. In den letzten Jahren ist die Bekanntheit von Python auch noch weiter gestiegen. Man geht also davon aus, dass man auch in Zukunft noch mit Python arbeiten kann.

% https://insights.stackoverflow.com/survey/2019#technology

\subsubsection*{Web Framework}
Bei Python selber steht man vor der Auswahl von verschiedenen Web Frameworks. Gemäss einer Umfrage von JetBrains über Python wird die Frage ''What web frameworks / libraries do you use in addition to Python?'' mit dem Kommentar ''Django and Flask continue to be by far the most popular Python web frameworks.'' zusammen gefasst. \\
Laut der oben genannten Umfrage verwenden 43\% der Entwickler  Django als Web Framework. Flask folgt dicht darauf mit 41\%. Das nächst bekannteste Framework ist Tornado mit nur 6\%. \\
Bei der Evaluation konzentrierte man sich deshalb hauptsächlich auf Django und Flask. Mit bekannteren Frameworks hat man den Vorteil, dass es einfacher ist, Hilfe und gute Dokumentationen zu finden. Mit weit verbreiteten Technologien kann man auch davon ausgehen, dass diese noch längere Zeit bestehen bleiben.  \\ 

Flask \cite{flask:overview} beschreibt sich selber mit den Worten:

''Flask is a lightweight WSGI web application framework. It is designed to make getting started quick and easy, with the ability to scale up to complex applications. It began as a simple wrapper around Werkzeug and Jinja and has become one of the most popular Python web application frameworks. \\
Flask offers suggestions, but doesn't enforce any dependencies or project layout. It is up to the developer to choose the tools and libraries they want to use. There are many extensions provided by the community that make adding new functionality easy.'' \\

Bei der Studienarbeit verwendete man bereits Flask zum Erstellen einer Webanwendung. Flask ist sicherlich ein gutes Framework, sonst würde es nicht von 41\% der Entwickler genutzt. Der grosse Vorteil von Flask ist, dass man als Entwickler relativ frei ist, wie man etwas umsetzen möchte. So kommt Flask standardmässig ohne Database Abstraction Layer, Form Validation oder anderen Tools, da bereits Libraries existieren, welche dies erledigen. Gerade am Anfang kann dies jedoch eher Verwirrung stiften, weil für das selbe Problem mehrere Lösungen bereit stehen. \\
% https://flask.palletsprojects.com/en/1.1.x/foreword/

Django verfolgt dagegen eher den Ansatz, bereits unterschiedliche Funktionalitäten mitzuliefern, was den Entwicklern Arbeit abnehmen soll. So ist zum Beispiel bereits ein OR-Mapper oder ein Admin Interface integriert. 
Django kommt aber auch mit einigen Nachteilen. Da einige Entscheidungen bereits vom Framework selber getroffen wurden, ist man als Entwickler nicht mehr ganz so flexibel. Zudem ist Django sehr monolithisch. Bei bestimmten Aufgaben ist vorgegeben, wie diese in Django umgesetzt werden müssen. Hält man sich nicht an den ''Django Way'', kann unter Umständen die gesamte Anwendung nicht deployed werden.

% https://docs.djangoproject.com/en/3.0/intro/overview/
% https://data-flair.training/blogs/django-advantages-and-disadvantages/

In der Studienarbeit konnte man zwar bereits Erfahrung mit Flask sammeln, dennoch entschied man sich für Django. Besonders das Django als ''Fully Loaded Framework'' und mit einem integrierten Admin Interface daher kommt, wurde als sehr positiv erachtet.

\subsubsection*{Datenbank}


\subsubsection*{Python Libraries}

\subsubsection{UI Design}
Bevor man mit der Entwicklung des Frontends beginnen konnte, musste geplant werden, wie das User Interface genau aussehen soll. Mit Hilfe der Use Cases konnte ein erster Entwurf erstellt werden. 

\subsection{Mockup}

\newpage



\subsection{Effektive Webseite}

\newpage
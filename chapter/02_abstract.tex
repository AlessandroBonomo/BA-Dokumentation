\afterpage{\blankpage}
\section*{Abstract}
\addcontentsline{toc}{section}{\protect\numberline{}Abstract}

Lehrende vermitteln ihren Schülern das Wissen aus spezifischen Schulfächern. Oftmals verändert sich dieses Wissen über die Jahre kaum, wie z.B. der Satz des Pythagoras. Lehrende verbringen viel Zeit damit, diesen Stoff den Lernenden beizubringen. Einige Lernende verstehen den Stoff problemlos, während schwächere Schüler Mühe haben und nicht richtig folgen können. \\

Es gibt zwar einige Programme oder Open-Source Projekte, welche Schülern den Schulstoff vermitteln sollen. Diese richten sich jedoch hauptsächlich direkt an die Schüler, indem zum Beispiel Nachhilfeunterricht für einzelne Fächer angeboten wird.\\

In dieser Bachelorarbeit wird eine Anwendung entwickelt, welche Schüler und Lehrer enger miteinander verbindet. Schüler können ein Thema auf der Plattform selbstständig und in ihrem eigenen Tempo erlernen. Lehrer stellen dazu die passenden Aufgaben zur Verfügung. Hat der Schüler Schwierigkeiten beim Lösen einer Aufgabe, so kann er mit Hilfestellungen Schritt für Schritt durch die Aufgabe geführt werden. Gibt es dennoch Unklarheiten, können allfällige Fragen in einem Forum gestellt werden, bei welchem sich alle Fragen rund um eine einzelne Aufgabe drehen. \\

Der Lehrer kann mit Hilfe von Statistiken jederzeit sehen, wie gut die Schüler ein spezifisches Thema verstehen. Somit muss der Lehrer nicht bis zur Prüfung warten, bis er den Wissensstand der Schüler sieht, sondern kann schon während der Lernphase schwächere Schüler gezielt unterstützen. Die anderen Schüler werden so auch nicht aufgehalten und können bereits mit dem nächsten Thema beginnen.




\newpage
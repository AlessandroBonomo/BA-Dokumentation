\afterpage{\blankpage}
\section*{Abstract}
\addcontentsline{toc}{section}{\protect\numberline{}Abstract}
Lehrer vermitteln ihren Schülern das Wissen aus spezifischen Schulfächern. Seit der Entdeckung des Satz des Pythagoras hat sich dieser jedoch nicht mehr verändert. Das bedeutet, dass Jahr für Jahr das selbe unterrichtet wird, was für die Lehrer schnell langweilig werden kann. \\

Es gibt zwar einige Programme oder Open-Source Projekte, welche Schülern den Schulstoff vermitteln wollen. Diese richten sich jedoch hauptsächlich direkt an die Schüler, indem zum Beispiel Nachhilfeunterricht für einzelne Fächer angeboten wird. \\

In dieser Bachelorarbeit wird eine Anwendung entwickelt, welche sich hauptsächlich auf die Schüler konzentriert. Die Schüler sind in der Lage, ein Thema selbstständig zu erlernen. Dazu stehen Theoriezusammenfassungen und Übungen zur Verfügung. Hat der Schüler Schwierigkeiten beim Lösen einer Aufgabe, so kann er mit Hilfestellungen Schritt für Schritt durch die Aufgabe geführt werden. Gibt es dennoch Unklarheiten, können alfällige Fragen in einem Forum gestellt werden, bei welchem sich alle Fragen rund um eine einzelne Aufgabe drehen. \\

Der Lehrer kann mit Hilfe von Statistiken jederzeit sehen, wie gut die Schüler ein spezifisches Thema verstehen. Dadurch muss der Lehrer nicht bis zur Prüfung warten um Feedback zu erhalten, sondern kann Wissenslücken bereits früh erkennen und darauf reagieren.

\newpage
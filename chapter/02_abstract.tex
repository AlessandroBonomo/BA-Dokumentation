\afterpage{\blankpage}
\section*{Abstract}
\addcontentsline{toc}{section}{\protect\numberline{}Abstract}
Die Lehrer vermitteln ihren Schülern das Wissen in den spezifischen Schulfächern. Seit der Erfindung des Satz von Pythagoras hat sich dieser aber nicht mehr geändert. Das bedeutet, dass Jahr für Jahr das selbe erzählt wird, was für die Lehrer sehr schnell langweilig werden kann. \\

Es gibt zwar einige Programme oder Open-Source Projekte, welche den Schulstoff an Schüler vermitteln wollen. Diese richten sich jedoch hauptsächlich an die Schüler direkt, indem zum Beispiel Nachhilfe für einzelne Fächer angeboten wird.  \\

In dieser Bachelorarbeit wird eine Anwendung entwickelt, welche sich hauptsächlich auf die Schüler konzentriert. Die Schüler sind in der Lage, ein Thema selbstständig zu erlernen. Dazu stehen Theorie Zusammenfassungen und Übungen zur Verfügung. Hat der Schüler Schwierigkeiten beim Lösen einer Aufgabe, so kann er mit Hilfestellungen Schritt für Schritt durch die Aufgabe geleitet werden. Gibt es trotzdem noch Unklarheiten, können alfällige Fragen in einem Forum gestellt werden, bei welchem sich alle Fragen rund um eine einzige Aufgabe drehen. \\

Der Lehrer kann mit Hilfe von Statistiken jederzeit sehen, wie gut die Schüler ein spezifisches Thema verstehen. So muss er nicht bis zu einer Prüfung warten um ein Feedback zu erhalten sondern kann die Wissenslücke der Schüler schnell füllen.

\newpage
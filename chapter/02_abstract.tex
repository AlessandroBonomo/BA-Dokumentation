\afterpage{\blankpage}
\section*{Abstract}
\addcontentsline{toc}{section}{\protect\numberline{}Abstract}

Die Schüler der Deutschschweizer Kantone werden mit der Einführung des Lehrplans 21 mit weitgehend gleichen Lerninhalten konfrontiert werden. Ein bestimmtes Thema wird dann also in der gleichen Art und Weise an eine wesentlich grössere Zahl von Schülern vermittelt. Gleichzeitig werden immer mehr Lerninhalte nach der Flipped-Classroom Methode vermittelt, wobei Schüler sich das Wissen weitgehend eigenständig erarbeiten und dabei von Lehrern betreut werden. Bei dieser Wissenserarbeitung spielen digitale Medien eine zentrale Rolle; die Rolle der Lehrenden wandelt sich dabei zu Coaches. \\

Es gibt bereits heute einige Programme oder Open-Source Projekte, welche auf diesen Trend aufspringen und Lernende beim Erarbeiten des Schulstoffs unterstützen wollen. Diese richten sich jedoch hauptsächlich direkt an Lernende, indem zum Beispiel Nachhilfeunterricht für einzelne Fächer angeboten wird. \\

In dieser Bachelorarbeit wird eine Anwendung entwickelt, welche Lernende und Lehrpersonen enger miteinander verbindet. Lernende können ein Thema auf der Plattform selbstständig und in ihrem eigenen Tempo erlernen. Lehrpersonen stellen dazu die passenden Aufgaben zur Verfügung. Hat der Lernende Schwierigkeiten beim Lösen einer Aufgabe, so kann er mit Hilfestellungen Schritt für Schritt durch die Aufgabe geführt werden. Gibt es dennoch Unklarheiten, können allfällige Fragen in einem Forum gestellt werden, bei welchem sich alle Fragen punktuell rund um diese spezifische Aufgabe drehen. \\

Die Lehrperson kann mit Hilfe von Statistiken jederzeit kontrollieren, wie gut die Lernenden ein spezifisches Thema verstehen. Somit muss die Lehrperson nicht bis zur Prüfung warten, um ein Feedback über den Wissensstand der Lernenden zu erhalten, sondern kann schon während der Lernphase gezielt auf jene Schüler eingehen, welche das Thema noch nicht ganz verstanden haben. Andere Schüler werden so auch nicht aufgehalten und können bereits mit dem nächsten Thema beginnen. Die Lehrperson behält so auch den Überblick über den Wissensstand der Lernenden und kann kontrollieren, dass alle Lernenden die einzelnen Themengebiete in der vorgegebenen Zeit abschliessen.


\newpage
\section{Architektur}
\subsection{System Übersicht}
Für die Umsetzung der Anwendung gibt es zwei Optionen. Man kann zwischen einer Native App, also einer klassische Client-Server Anwendung, oder eine Web Anwendung, entscheiden. \\

Native Apps haben zwar einige Vorteile. Die Integration mit anderen Anwendungen ist so zum Beispiel einfacher. Zudem wäre die Monetarisierung auch um einiges einfacher, denn eine Native App kann sehr gut im App Store platziert werden. So steigt auch gleich die Bekanntheit der Applikation\footcite{native_app}. \\

Die Entscheidung bei der Entwicklung von Aufgaben-Coaching fiel jedoch sehr schnell auf eine Web Anwendung. Viele Schulen haben zwar Apple iPads im Einsatz. Trotzdem kann es Schulen geben, welche mit Android Tablets arbeiten. Hinzu kommt noch die Anforderung der Lehrer. Da diese hauptsächlich auf Notebooks oder Desktop PCs arbeiten, müsste für sie auch eine separate Anwendung entwickelt werden. Mit einer Webanwendung hat man den Vorteil, nur eine Anwendung entwickeln zu müssen, auf welche alle Benutzer Zugriff haben. Für grosse Firmen ist dies kein Problem, mehrere Anwendungen für unterschiedliche Benutzergruppen zu entwickeln. Möchte man die Idee aber als Startup weiter verfolgen, hat man zu Beginn weder die Zeit, noch das Kapital, um gleich mehrere Anwendungen zu entwickeln. \\

Zusätzlich kommt noch die Anforderung der Lehrer und Admins hinzu. Im Gegensatz zu den Schülern arbeiten diese meistens an einem Notebook

Zudem kann man mit Web Anwendungen die Supportkosten relativ tief halten, was auf Schulen vielleicht noch ein bisschen attraktiver wirkt. Die Anwendung kann direkt im Browser aufgerufen werden und es muss nichts im vorhinein auf jedem Device installiert werden.

\begin{figure}[H]
\begin{center}
	\includegraphics[width=\textwidth, keepaspectratio]{images/system_overview.png}
	\caption{System Overview}
	\label{fig:system_overview}
\end{center}
\end{figure}


\subsection{Domain Model}
In der Abbildung \ref{fig:domain_model_full} ist das vollständige Domain Model Model abgebildet. Dem Leser soll so ein Überblick über die einzelnen Teilgebiete der Anwendung gegeben werden.

\begin{landscape}

\begin{minipage}{\textwidth}
	\begin{figure}[H]
	\begin{center}
		\includegraphics[width=1.4\textwidth, keepaspectratio]{images/domain_model_full.png}
  		\caption{Vollständiges Domain Model}
		\label{fig:domain_model_full}
	\end{center}
	\end{figure}
\end{minipage}

\end{landscape}

\subsubsection*{School}
Ein zentraler Punkt der Anwendung ist die Schule und das User Management. Die User sind einer der drei Rollen Student, Teacher oder Admin zugeteilt. Da jede Rolle unterschiedliche Rechte hat, konnte nicht das Standard User Model von Django selber verwendet werden. \\
Es stehen drei Möglichkeiten zur Verfügung, wie unterschiedliche User Rollen implementiert werden können.

\begin{enumerate}
	\item Unterscheidung der User anhand eines Boolean Flags
	\item Unterscheidung der User anhand eines Choices Field 
	\item Erstellung eines Profile Models mit einer 1 zu 1 Beziehung zum User Model
\end{enumerate}

Die erste Variante hat den Vorteil, dass ein User mehrere Rollen gleichzeitig haben kann. Im Gegensatz zur ersten Variante kann ein User bei der zweiten Variante nur eine Rolle annehmen kann. Diese beiden Varianten haben jedoch den Nachteil, dass jeder User die gleichen Daten hat. Möchte man einem Schüler eine Matrikelnummer zuweisen, existiert dieses Feld implizit auch für alle Lehrer und Admins. \\
Um diesem Problem aus dem Weg zu gehen, entschied man sich für die dritte Variante. Pro Rolle existiert ein Profile Model. Möchte man nur einer bestimmten Rolle ein Feld zuweisen, so kann dieses Feld im entsprechenden Profile Model hinzugefügt werden\footcite{django:user_types}. \\

\begin{figure}[H]
	\begin{center}
	\includegraphics[width=\textwidth, keepaspectratio]{images/domain_model_user.png}
	\caption{Domain Model - User Abschnitt}
	\label{fig:domain_model_user}
	\end{center}
\end{figure}

Der Admin kann als einziger User Typ für sich alleine existieren. Schüler und Lehrer werden Klassen zugewiesen. Ein Lehrer kann gleichzeitig mehreren Klassen zugewiesen sein, ein Schüler kann sich jedoch nur in einer Klasse befinden. Für den aktuellen Stand reicht diese Funktionalität aus. Es gibt jedoch den Use Case, in welchem Schüler aus unterschiedlichen Klassen Fächer zusammen belegen. Dies ist zum Beispiel der Fall, wenn Migrationskinder aus unterschiedlichen Klassen zusammen kommen und gemeinsam ein Deutsch Nachhilfe Fach besuchen. Für diesen Fall kann man dem Student Profile ein Boolean Flag zuweisen. Falls dieses Flag gesetzt ist, hat der Schüler Zugriff auf Nachhilfe Fächer. \\

Wenn dieses Projekt als Startup weiter verfolgt wird, dann würde pro Schule eine neue Instanz der Anwendung deployed werden. So entstehen keine Multi Tenancy Probleme. Möchte man in Zukunft jedoch einen Multi Tenancy Ansatz verfolgen, müsste das Domain Model dementsprechend angepasst werden. Ein möglicher Lösungsansatz wäre, wenn jeder Schule eine eigene Subdomain, wie zum Beispiel schule-gossau.aufgaben-coach.ch, vergeben würde. Anhand der Subdomain können dann die einzelnen User identifiziert werden\footcite{django:multi_tenancy}.

\newpage
\subsubsection*{Study Content}
Ein zentraler Punkt dieser Anwendung sind die Schulfächer. Pro Subject gibt es mehrere Topics. Topics wiederum können mehrere Lessons haben. \\ 

\begin{figure}[H]
\begin{center}
	\includegraphics[width=\textwidth, keepaspectratio]{images/domain_model_study_content.png}
	\caption{Domain Model - Schulfächer}
	\label{fig:domain_model_study_content}
\end{center}
\end{figure}


Ein Subjects kann einer Klasse zugewiesen werden, welche erst dann Zugriff auf die Topics und Lessons hat. Im Moment ist es nicht möglich, einzelne Topics oder Lessons freizuschalten. Möchte man diese Funktionalität in Zukunft haben, muss eine neue Tabelle, ähnlich wie SubjectEnrollment, erstellt werden.

\newpage
\subsubsection*{Exercise}
Das Herzstück der Anwendung sind die Exercises. Lehrer können Übungen erstellen, welche aus mehreren Teilaufgaben bestehen, welche wiederm mehrere Hilfestellungen haben. 

\begin{figure}[H]
\begin{center}
	\includegraphics[width=\textwidth, keepaspectratio]{images/domain_model_exercise.png}
	\caption{Domain Model - Aufgaben}
	\label{fig:domain_model_exercise}
\end{center}
\end{figure}

Die Tabelle Answer dient zur Speicherung der Antworten von Schülern. Die beiden Felder ''is\_submitted'' und ''is\_archived'' sind für den Status der Aufgaben. Ist eine Aufgabe submitted, also abgegeben, kann sie nicht weiter vom Schüler bearbeitet werden. Erst wenn Aufgaben abgegeben sind, können sie vom Lehrer angeschaut werden. \\ 
Wurden die Aufgaben vom Lehrer korrigiert, bekommen sie den Status ''is\_archived''.

\newpage
\subsubsection*{Forum}
Für jede Frage in einer Übung gibt es ein Forum. Die Schüler haben einen Ort, in welchem Fragen gestellt werden können. Da es für jede Frage ein Forum gibt, werden die Schüler auch nicht von anderen Forumsbeiträgen abgelenkt.
\begin{figure}[H]
\begin{center}
	\includegraphics[width=0.3\textwidth, keepaspectratio]{images/domain_model_forum.png}
	\caption{Domain Model - Forum}
	\label{fig:domain_model_forum}
\end{center}
\end{figure}


\subsection{Tools und Frameworks}
Bei der Wahl von Technologien, Tools und Frameworks achtete man besonders darauf, dass die entsprechenden Tools auch in der Zukunft noch supported werden. Dazu wurde geschaut, ob und wie gross die Community hinter einzelnen Tools ist und wie gut die Tools dokumentiert sind.

\subsubsection*{Programmiersprache}
Für das Web Development gibt es mehrere Programmiersprachen, die zur Auswahl standen. Die Wohl bekanntesten darunter sind:

%https://en.wikipedia.org/wiki/Web_framework

\begin{itemize}
	\item Python (Django / Flask)
	\item C\# (ASP.NET Core)
	\item Java (Spring)
	\item JavaScript (Express.js / Node.js / Sails.js)
	\item PHP (CakePHP / CodeIgniter / Laravel)
	\item Perl (Catalyst / Mojolicious)
	\item Ruby (Ruby on Rails)
\end{itemize}

%TODO Satzstellung anschauen

Die Wahl fiel schnell auf Python, da bereits etwas KnowHow über diese Programmiersprache vorhanden ist. \\
Gemäss Developer Survey Results\footcite{developer_survey_results} von Stack Overflow ist Python (41.7\%) etwas bekannter als Java (41.1\%). Nur JavaScript (67.8\%) ist noch bekannter. In den letzten Jahren ist die Bekanntheit von Python auch noch weiter gestiegen. Man geht also davon aus, dass man auch in Zukunft noch mit Python arbeiten kann.


\subsubsection*{Web Framework}
Bei Python selber steht man vor der Auswahl von verschiedenen Web Frameworks. Gemäss einer Umfrage von JetBrains über Python wird die Frage ''What web frameworks / libraries do you use in addition to Python?'' mit dem Kommentar ''Django and Flask continue to be by far the most popular Python web frameworks.'' zusammen gefasst. \\
Laut der oben genannten Umfrage verwenden 43\% der Entwickler  Django als Web Framework. Flask folgt dicht darauf mit 41\%. Das nächst bekannteste Framework ist Tornado mit nur 6\%. \\
Bei der Evaluation konzentrierte man sich deshalb hauptsächlich auf Django und Flask. Mit bekannteren Frameworks hat man den Vorteil, dass es einfacher ist, Hilfe und gute Dokumentationen zu finden. Mit weit verbreiteten Technologien kann man auch davon ausgehen, dass diese noch längere Zeit bestehen bleiben.  \\ 

%TODO zitat einbauen
Flask\footcite{flask:foreword} beschreibt sich selber mit den Worten:

\begin{displayquote}
Flask is a lightweight WSGI web application framework. It is designed to make getting started quick and easy, with the ability to scale up to complex applications. It began as a simple wrapper around Werkzeug and Jinja and has become one of the most popular Python web application frameworks. \\
Flask offers suggestions, but doesn't enforce any dependencies or project layout. It is up to the developer to choose the tools and libraries they want to use. There are many extensions provided by the community that make adding new functionality easy.'
\end{displayquote}

Bei der Studienarbeit verwendete man bereits Flask zum Erstellen einer Webanwendung. Flask ist sicherlich ein gutes Framework, sonst würde es nicht von 41\% der Entwickler genutzt. Der grosse Vorteil von Flask ist, dass man als Entwickler relativ frei ist, wie man etwas umsetzen möchte. So kommt Flask standardmässig ohne Database Abstraction Layer, Form Validation oder anderen Tools, da bereits Libraries existieren, welche dies erledigen\footcite{flask:design}. Gerade am Anfang kann dies jedoch eher Verwirrung stiften, weil für das selbe Problem mehrere Lösungen bereit stehen. \\

Django verfolgt dagegen eher den Ansatz, bereits unterschiedliche Funktionalitäten mitzuliefern, was den Entwicklern Arbeit abnehmen soll. So ist zum Beispiel bereits ein OR-Mapper oder ein Admin Interface integriert\footcite{django:overview}. 
Django kommt aber auch mit einigen Nachteilen. Da einige Entscheidungen bereits vom Framework selber getroffen wurden, ist man als Entwickler nicht mehr ganz so flexibel. Zudem ist Django sehr monolithisch. Bei bestimmten Aufgaben ist vorgegeben, wie diese in Django umgesetzt werden müssen. Hält man sich nicht an den ''Django Way'', kann unter Umständen die gesamte Anwendung nicht deployed werden\footcite{django:advantages_disadvantages}.

In der Studienarbeit konnte man zwar bereits Erfahrung mit Flask sammeln, dennoch entschied man sich für Django. Besonders das Django als ''Fully Loaded Framework'' und mit einem integrierten Admin Interface daher kommt, wurde als sehr positiv erachtet.


\subsubsection*{Datenbank}
Um ein Datenbanksystem kommt diese Anwendung nicht herum. Folgende Überlegungen wurden sich bei der Evaluation für eine geeignete Datenbank gemacht:

\begin{itemize}
	\item Die Datenbank muss skalierbar sein, so dass sie auch eine grosse Anzahl Benutzer unterstützt.
	\item Da ein Grossteil der Benutzer mit Mobile Devices auf die Anwendung zugreift, soll bei Anfragen möglichst wenig Overhead geliefert werden.
	\item Die Schreib- und Lese-Operationen sind wichtig, jedoch nicht kritisch. 
\end{itemize}

Aufgrund der ohnehin schon verkürzten Projektlaufzeit entschied man sich, auf die Evaluation eines Datenbanksystems zu verzichten. Während den Modulen Datenbanksysteme 1 \& 2 konnte bereits sehr viel Erfahrungen mit Postgres gesammelt werden. Laut Deployment Statistics\footcite{deploymentstatistics} von Django Sites verwenden 47.7\% aller deployten Sites MySQL als Datenbank. Postgres folgt mit 40.7\%. Das nächst grössere Datenbanksystem ist sqlite mit nur noch 7.6\%.
MySQL wird zwar etwas öfters verwendet. Aus Erfahrung weiss man aber, dass die Funktionen von Postgres die Bedürfnisse dieser Anwendung voll und ganz abdecken, weshalb man Postgres als Datenbanksystem festlegte. \\

Sollte man in Zukunft jedoch ein anderes Datenbanksystem bevorzugen, kann die Migration mit dem Django Management Tool durchgeführt werden \footcite{dbmigration}. 

\subsubsection*{Python Libraries}
Libraries beschreiben


\subsection{Deployment}
Zusätzlich zur System Overview in Abbildung \ref{fig:system_overview}, wird hier das Deployment Diagramm gezeigt. Dieses soll die einzelnen Komponenten aufzeigen.

\begin{figure}[H]
\begin{center}
	\includegraphics[width=\textwidth, keepaspectratio]{images/deployment_diagram.png}
	\caption{Deployment Diagramm}
	\label{fig:deployment_diagram}
\end{center}
\end{figure}

\subsubsection*{Mailserver}
Warum braucht es den Mailserver?

\subsubsection*{Webserver}
Apache oder NGINX

\subsection*{WSGI}
was ist das? warum gunicorn?


\subsection{Projektplan}
Während der Bachelorarbeit wurde der Projektplan als eigenständiges Dokument geführt. Falls Änderungen vorgekommen sind, wurde der Projektplan dementsprechend angepasst. Gegen Ende der Bachelorarbeit wurde der Projektplan in die Dokumentation eingefügt

\subsubsection{Inhalt}
\subsubsection*{Zweck}
Dieses Dokument beschreibt die Planung der Bachelorarbeit, in welchem eine Lernplattform für Schulen entwickelt wird.

\subsubsection*{Gültigkeitsbereich}
Dieses Dokument ist während der gesamten Laufzeit der Bachelorarbeit gültig. Die Änderungsgeschichte kann in Github nachverfolgt werden.

\subsubsection*{Referenzen}
Dieses Dokument wurde mit dem Wissen erstellt, welches in den Modulen Software Engineering 1 \& 2, der Studienarbeit sowie in gewissen Grenzen in Cloud Infrastructure und Web Engineering \& Design, vermittelt wird.



\subsubsection{Projektübersicht}
Bei dieser Bachelorarbeit wird eine Lernplattform erstellt, mit welchem Lehrer und Schüler zusammenarbeiten können. Auf dieser Plattform kann die Theorie der einzelnen Fächer in Form von Videos oder Theoriezusammenfassungen vermittelt werden. Zudem können Übungen erstellt und zur Verfügung gestellt werden. \\
Der Lehrer kann eigene Aufgaben erfassen, welche die Schüler lösen können. Nachdem die Aufgaben gelöst wurden, sieht der Lehrer eine Statistik, ob und wie gut die einzelnen Aufgaben gelöst wurden. Falls eine Aufgabe überdurchschnittlich schlecht gelöst wurde, kann er diese direkt im Unterricht ansprechen und allfällige Fragen klären. \\
Die Kernaufgabe dieser Bachelorarbeit ist der aufgabenspezifische Teil. Ferner kann die Plattform aber noch um zusätzliche Features wie einen Chat oder ein Forum erweitert werden.


\subsubsection*{Lieferumfang}
Folgende Dokumente werden am Ende der Bachelorarbeit abgeliefert:
\begin{itemize}
	\itemsep0em
	\item Abstract
	\item Aufgabenstellung
	\item Projektplan
	\item Projektdokumentation
	\item Installations Anleitung
	\item Bedienungs Anleitung
	\item Einverständniserklärung
	\item Erklärung zur Urheberschaft
	\item Passwörter
	\item Persönliche Berichte
	\item Protokolle
	\item Source Code	
\end{itemize}



\subsubsection{Organisation}
\subsubsection*{Personen und Rollen}
Nachfolgend werden alle an dieser Bachelorarbeit beteiligten Personen und ihre Rollen aufegelistet.
\subsubsection*{Luca Gubler}
Die Idee für das Thema dieser Bachelorarbeit stammt von Luca. Aus diesem Grund kümmert er sich um den groben Projektverlauf der Arbeit. Bei der Arbeit selber kümmert er sich hauptsächlich um die Infrastruktur, die Datenbank und das Backend der Applikation.

\subsubsection*{Alessandro Bonomo}
Alessandro kümmert sich hauptsächlich um das Frontend und das UI der Applikation. Er arbeitet jedoch auch am Backend mit und kümmert sich um das Testing der Applikation.

\subsubsection*{Externe Personen}
Bei dieser Bachelorarbeit übernimmt Professor Frank Koch die Rolle des Betreuers. Stephan Meier übernimmt die Rolle als externer Co-Examinators. Zusätzlich wird Professor Laurent Metzger diese Bachelorarbeit als interner Co-Examinator betreuen.

\newpage
\subsubsection{Management Abläufe}
\subsubsection*{Zeitbudget}
Die Bachelorarbeit begann in der Woche vom  16. September 2019 und dauert insgesamt 17 Wochen. Für das Erreichen der 12 ECTS ist geplant, dass jedes Teammitglied 360 Stunden arbeitet. Daraus resultiert eine durchschnittliche Arbeitszeit von knapp 24 Stunden pro Woche.

\begin{table}[h]
	\centering
	\begin{tabu} to 0.9\textwidth {l X}
	\toprule
	Projektstart & 13.10.2019 \\
	Projektdauer & 13 Wochen \\
	Arbeitsstunden pro Person & 23h pro Woche, Total 300h \\
	Arbeitsstunden Total & 600h \\
	Projektende & 10.01.2020 \\ 
	\bottomrule
	\end{tabu}
	\captionof{table}{Übersicht Zeitaufwand}
\end{table}


\noindent Für die Bachelorarbeit stehen total 720 Stunden zur Verfügung. Da jedoch für das erste Thema ca. 60 Arbeitsstunden pro Person aufgewendet wurden, stehen für die neue Arbeit noch total 600 Stunden zur Verfügung. Mit dem definierten Projektumfang wird diese Arbeitszeit voraussichtlich vollständig ausgenutzt. Sollte der Umfang jedoch früher als erwartet abgeschlossen werden können, kann das Projekt um weitere Funktionalitäten erweitert werden.

\subsubsection*{Projektmanagement}
Als Projektmanagement Methode wurde SCRUM+ gewählt. Bei dieser Projektmanagement Methode handelt es sich um einen Mix aus SCRUM und Unified Process. Diese Methode wird auch von Daniel Keller im Modul ''Software Engineering'' unterrichtet.

\subsubsection*{Phasen / Sprints}
Bei SCRUM+ wird das gesamte Projekt in die vier Phasen Inception, Elaboration, Construction und Transition eingeteilt. Pro Phase gibt es wiederum einzelne Sprints. Zudem wurden einzelne Meilensteine definiert, welche auf der untenstehenden Tabelle entnommen werden können.

\subsubsection*{Iterationsplanung}
Die Dauer eines Sprints wurde auf 2 Wochen festgelegt. Zu Beginn jedes Sprints setzt sich das Team zusammen um den nächsten Sprint zu planen. Dabei wird jeweils besprochen, welche Aufgaben des vergangenen Sprints nicht vollständig abgeschlossen werden konnten. Die nicht abgeschlossenen Arbeiten werden mit neu definierten Aufgaben in den neuen Sprint übernommen und jeweils zeitlich abgeschätzt und priorisiert. Da sich im Team nur 2 Mitglieder befinden, wird darauf verzichtet, die Arbeitspakete unter den Teammitgliedern untereinander zuzuweisen. Die gesamte Planung und Verwaltung der Aufgaben wird in Jira erledigt.


\newpage
\subsubsection*{Meilensteine}
Folgende Meilensteine wurden gesetzt. Da der Projektumfang aber relativ gross war und durch den Wechsel der Projektarbeit weniger Zeit zur Verfügung stand, konnten nicht alle Meilensteine eingehalten werden. \\
Der Meilenstein ''Feature Freeze'' musste um 2 Wochen nach hinten geschoben werden. Die Zeit reichte schlicht und einfach nicht aus, um alle geforderten Funktionalitäten in der vorgenommenen Zeit umzusetzen. 

%TODO cell spacing
\begin{table}[h]
	\centering
	\begin{tabu} to 0.9\textwidth {l l X}
	\toprule
	SW & Meilenstein & Beschreibung \\ 
	\midrule
	4 & M0: Kickoff & Start des Projektes \\ 
	5 & M1: Abschluss Projektplan & Projektplan erstellt und mit Betreuer besprochen \\
	7 & M2: End of Elaboration & Use Cases und funktionale sowie nicht funktionale Anforderungen sind erfasst. Mockups Domainanalyse und Konzept für die Architektur sind erstellt. \\
	12 & M3: Feature Freeze & Entwicklung der Features ist abgeschlossen, damit man sich auf Bugfixes und Code Qualität konzentrieren kann.\\
	15 & M4: Code Freeze & Entwicklung an der Applikation ist abgeschlossen. End of Construction.\\
	17 & M5: Projektende & Abgabe der Bachelorarbeit \\ 
	\bottomrule
	\end{tabu}
	\captionof{table}{Übersicht Meilensteine}
\end{table}

\subsubsection*{Termine}
Folgende Termine sind von der HSR oder dem Betreuer vorgegeben und müssen eingehalten werden.

\begin{table}[h]
	\centering
	\begin{tabu} to 0.9\textwidth {l l X}
	\toprule
	Datum & Beschreibung \\ 
	\midrule
	6.1.1010 & Erfassung des Abstracts im Online-Tool https://abstract.hsr.ch/ Die
Studierenden geben den Abstract für die Diplomarbeitsbroschüre zur
Kontrolle an ihren Betreuer/Examinator frei.
Vorlagen sowie eine ausführliche Anleitung betreffend Dokumentation
stehen auf dem Skripteserver zur Verfügung.
Der Betreuer/Examinator gibt das Dokument mit dem korrekten und
vollständigen Abstract der Broschüre zur Weiterverarbeitung an das
Studiengangsekretariat frei. \\
	10.1.2020 & Abgabe des Berichts an den Betreuer und Hochladen aller Dokumente auf archiv-i.hsr bis 17 Uhr \\
	10.2.2020 & Präsentation der Bachelorarbeit und mündliche Prüfung.
	\bottomrule
	\end{tabu}
	\captionof{table}{Übersicht Termine}
\end{table}

\subsection*{Projektverwaltung}
%TODO JIRA workflow beschreiben
Als Projektverwaltungstool wird Jira verwendet. Man hat sich für dieses Tool entschieden, da es die gewünschte Funktionalität mit sich bringt. Um den Stand der einzelnen Arbeitspakete möglichst genau darzustellen, wurde ein Workflow definiert, welcher jedes Arbeitspaket durchlaufen muss.
%TODO BILD Workflow einfügen
%TODO ich wür review als workflow schritt use neh. wenn du arbeitspaket erfassisch, denn mussi das ned nomal aluege. review isch guet für code. wenn du doku korrigiersch, wottis ned nomal müsse aluege.
Wie im Bild ersichtlich ist, muss jedes Arbeitspaket folgende Status durchlaufen: Open, In Progress, Review und Done. Ein Arbeitspaket muss jeden Schritt im Workflow durchlaufen, ausser Review. Dieser Schritt ist optinal und muss nicht bei jedem Arbeitspaket durchgeführt werden.

\subsection*{Zeiterfassung}
Die Zeiterfassung wird mit Jira verwaltet. Beim Einfügen von Arbeitspaketen in den Sprint wird die Zeit geschätzt, welche für das Arbeitspaket aufgewendet werden muss. Jede Person, welche an diesem Arbeitspaket gearbeitet hat, kann Zeit auf dieses Arbeitspaket buchen. Am Schluss kann so eine Zeitauswertung über die einzelnen Sprints oder das gesamte Projekt gemacht werden.

\subsubsection*{Meetings}

\subsubsection{Risikomanagement}

\subsubsection*{Risiken}

\subsubsection*{Massnahmen}

\subsubsection*{Auswertung}

\subsubsection{Qualitätsmanagement}

\subsubsection*{Dokumentation}

\subsubsection*{Projektmanagement}

\subsubsection*{Entwicklung}

\subsubsection*{Code Reviews}

\subsubsection{Infrastruktur}

\newpage
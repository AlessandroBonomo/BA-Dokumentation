\section{Technischer Bericht}
\subsection{Einleitung}

\subsubsection{Hintergrund}
Luca Gubler arbeitete während neben dem Studium als IT Supporter in der Oberstufe Gossau ZH. Während eines Gespräches mit seinem Vorgesetzten bemängelte dieser, dass es keine guten E-Learning Plattformen für Schulen gibt. Zwar gebe es vereinzelt Lehrer oder Schulen, welche dies selber in die Hand nehmen und eine Open Source Tool wie Moodle verwenden und darauf den Content selber erfassen. Es braucht jedoch relativ viel Zeit, bis der gesamte Schulstoff erfasst ist. Die Qualität lässt meistens aber zu wünschen übrig.

\subsubsection{Problemstellung / Vision}
Aus diesem Gespräch heraus entstand die Idee, eine All-In-One Lernplattform für Schulen zu erstellen. Diese Plattform soll den ganzen Schulstoff der Sekundarstufe 1 in Form von Videos und Theorie Zusammenfassungen enthalten. Zudem gibt es Quizze und Übungen zu den einzelnen Themen. Sobald die Schüler eine Aufgabe oder ein Quiz gelöst haben, kann der Lehrer Statistiken einsehen, wie gut die Schüler diese Aufgaben gelöst haben. Falls zum Beispiel eine Aufgabe besonders schlecht gelöst wurde, kann der Lehrer diese Aufgabe mit der ganzen Klasse besprechen. Zudem kann der Lehrer auf einzelne Schüler zugehen, falls er bemerkt, dass diese Hilfe in einem speziellen Aufgabengebiet benötigen.
\\
Des Weiteren soll ein Forum bereitstehen, in welchem Schüler Fragen zu spezifischen Aufgaben stellen können. Der Lehrer bekommt eine Meldung, falls seine Schüler Fragen gestellt haben und kann diese auch gleich selber beantworten.

\subsubsection{Aufgabenstellung}
Bei dieser Problemstellung handelt es sich um eine grobe Beschreibung der gesamten Idee. Da der Umfang dieser Bachelorarbeit begrenzt ist, können nicht alle Punkte umgesetzt werden. Das grösste Problem ist, dass diese Lernplattform an sich relativ ähnlich zu Moodle ist. Aus diesem Grund muss man sich davon abheben können.
\\
In Zusammenarbeit mit Frank Koch, dem Betreuer dieser Bachelorarbeit und dem Moodle Experten der HSR kam man zum Schluss, dass man sich mit dem aufgabenspezifichen Teil der Anwendung von Moodle abgrenzen kann. Schüler können eine Aufgabe lösen. Fall diese jedoch nicht wissen, wie man genau vorgehen muss, können sie Hilfe anfordern. Anschliessend werden sie Schritt für Schritt durch die Aufgabe geleitet und kommen so zum Ziel.
\\
Schlussendlich soll jedoch eine lauffähige Anwendung existieren, also kommt einiges an Funktionalität hinzu. So muss zum Beispiel ein User Management existieren. Ein Administrator muss neue Lehrer und Schüler erfassen und diese einer Schulklasse zuweisen können.
\\
Auf der Lernplattform ist der gesamte Schulstoff der Sekundarstufe vorhanden. Die Lehrer können einzelne Fächer freischalten, so dass diese für die Schüler sichtbar sind. Haben die Schüler eine Aufgabe gelöst, kann der Lehrer sehen, wie viele Schüler diese Aufgabe gelöst haben und wie gut der Klassenschnitt ist.
\\
Die Schüler haben die Möglichkeit, die freigeschaltenen Fächer anzusehen und Quizze und Übungen zu lösen.

\subsubsection{Zielgruppen}
In einem ersten Schritt richtet sich der Lerncoach an alle drei Stufen der Sekundarstufe 1. Später soll der Lerncoach aber auch für die Primarstufe oder Sekundarstufe 2 zugänglich gemacht werden. 

\subsection{Stand der Technik}
Es gibt bereits einige Tools, welche Lerninhalte für Schüler zur Verfügung stellen. Diese bestehenden Tools lassen sich in zwei Kategorien einteilen:
\begin{itemize}
	\item Tools von kommerziellen Anbietern
	\item Open-Source Tools
\end{itemize}

\subsubsection{Kommerzielle Anbieter}
Sofatutor gehört wohl zu den grössten und bekanntesten Anbietern von Lerninhalten im DACH Raum. Sofatutor richtet sich jedoch hauptsächlich an Schüler, welche Nachhilfe in einem bestimmten Fach brauchen. Es ist zwar möglich, sofatutor im Schulalltag einzusetzen. Lehrer können dann Videos freischalten, welche die Schüler sehen können. \\

Mit dem Lerncoach verfolgt man aber das Ziel des ''Flipped Classroom''. 

\newpage
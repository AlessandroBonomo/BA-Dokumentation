\section{Technischer Bericht}
\subsection{Einleitung}

\subsubsection{Hintergrund}
Luca Gubler arbeitete während und neben dem Studium als IT Supporter in der Oberstufe Gossau ZH. Während eines Gespräches mit seinem Vorgesetzten bemängelte dieser, dass es keine guten E-Learning Plattformen für Schulen gibt. Zwar gebe es vereinzelt Lehrer oder Schulen, welche dies selber in die Hand nehmen und eine Open Source Tool wie Moodle verwenden und darauf den Content selber erfassen. Es braucht jedoch relativ viel Zeit, bis der gesamte Schulstoff erfasst ist. Auch die Qualität lässt meistens aber zu wünschen übrig.

\subsubsection{Problemstellung / Vision}
Aus diesem Gespräch heraus entstand die Idee, eine All-In-One Lernplattform für Schulen zu erstellen. Diese Plattform soll den ganzen Schulstoff der Sekundarstufe 1 in Form von Videos und Theorie-Zusammenfassungen enthalten. Zudem gibt es Quizze und Übungen zu den einzelnen Themen. Sobald die Schüler eine Aufgabe oder ein Quiz gelöst haben, kann der Lehrer Statistiken einsehen, wie gut die Schüler diese Aufgaben gelöst haben. Falls zum Beispiel eine Aufgabe besonders schlecht gelöst wurde, kann der Lehrer diese Aufgabe mit der ganzen Klasse besprechen. Zudem kann der Lehrer auf einzelne Schüler zugehen, falls er bemerkt, dass diese Hilfe in einem speziellen Aufgabengebiet benötigen.
\\
Des Weiteren soll ein Forum bereitstehen, in welchem Schüler Fragen zu spezifischen Aufgaben stellen können. Der Lehrer bekommt eine Meldung, falls seine Schüler Fragen gestellt haben und kann diese auch gleich selber beantworten.

\subsubsection{Aufgabenstellung}
Bei dieser Problemstellung handelt es sich um eine grobe Beschreibung der gesamten Idee. Da der Umfang dieser Bachelorarbeit begrenzt ist, können nicht alle Punkte umgesetzt werden. Das grösste Problem ist, dass diese Lernplattform an sich relativ ähnlich zu Moodle ist. Aus diesem Grund muss man sich davon abheben können.
\\
In Zusammenarbeit mit Frank Koch, dem Betreuer dieser Bachelorarbeit und dem Moodle Experten der HSR kam man zum Schluss, dass man sich mit dem aufgabenspezifichen Teil der Anwendung von Moodle abgrenzen kann. Schüler können eine Aufgabe lösen. Fall diese jedoch nicht wissen, wie man genau vorgehen muss, können sie Hilfe anfordern. Anschliessend werden sie Schritt für Schritt durch die Aufgabe geleitet und kommen so zum Ziel.
\\
Schlussendlich soll jedoch eine lauffähige Anwendung existieren, also kommt einiges an Funktionalität hinzu. So muss zum Beispiel ein User Management existieren. Ein Administrator muss neue Lehrer und Schüler erfassen und diese einer Schulklasse zuweisen können.
\\
Auf der Lernplattform ist der gesamte Schulstoff der Sekundarstufe vorhanden. Die Lehrer können einzelne Fächer freischalten, so dass diese für die Schüler sichtbar sind. Haben die Schüler eine Aufgabe gelöst, kann der Lehrer sehen, wie viele Schüler diese Aufgabe gelöst haben und wie gut der Klassenschnitt ist.
\\
Die Schüler haben die Möglichkeit, die freigeschaltenen Fächer anzusehen und Quizze und Übungen zu lösen.

\subsubsection{Zielgruppen}
In einem ersten Schritt richtet sich der Lerncoach an alle drei Stufen der Sekundarstufe 1. Später soll der Lerncoach aber auch für die Primarstufe oder Sekundarstufe 2 zugänglich gemacht werden. 

\subsection{Stand der Technik}
Es gibt bereits einige Tools, welche Lerninhalte für Schüler zur Verfügung stellen. Diese bestehenden Tools lassen sich in zwei Kategorien einteilen:
\begin{itemize}
	\item Tools von kommerziellen Anbietern
	\item Open-Source Tools
\end{itemize}

\subsubsection{Kommerzielle Anbieter}
Sofatutor gehört wohl zu den grössten und bekanntesten Anbietern von Lerninhalten im DACH Raum. Sofatutor richtet sich jedoch hauptsächlich an Schüler, welche Nachhilfe in einem bestimmten Fach brauchen. Es ist jedoch möglich, Sofatutor im Schulalltag einzusetzen. Lehrer können Videos freischalten, welche die Schüler dann sehen können. \\

\noindent Mit dem Aufgaben-Coach verfolgt man aber das Ziel des ''Flipped Classroom''. Bei der klassischen Methode, wie sie zur Zeit in der Schule angewendet wird, lernen die Schüler die Theorie in der Schule und vertiefen das Wissen durch Übungen zu Hause. Beim ''flipped classroom'' stellt der Lehrer den Schülern die Theorie zum Beispiel als Video zur Verfügung. Die Schüler können die Theorie so zu Hause lernen und in der Schule dann die darauf aufbauenden Übungen lösen. So benötigen die Schüler nur dann die Hilfe des Lehrers, wenn sie vor einem Problem stehen. Der Lehrer steht also nur noch als eine Art Coach zur Verfügung. 

\subsubsection{Open Source}
Moodle ist die wahrscheinlich bekannteste Open Source LMS. Bei Moodle handelt es sich aber nur um die Plattform an sich. Im Gegensatz zu kommerziellen Lösungen wird hier kein Content zur Verfügung gestellt. Neben dem Content braucht es aber auch noch einen Verantwortlichen an der Schule, welcher sich um die Verwaltung von Moodle kümmert. \\

\noindent Auf den ersten Blick hat der Aufgaben-Coach sehr viele Ähnlichkeiten zu Moodle. In Moodle ist es zwar möglich, Aufgaben zu erstellen und auszuwerten, eine enge Betreuung der Schüler ist jedoch nicht direkt möglich. \\

\noindent An diesem Punkt möchte man mit dem Aufgaben-Coach ansetzen. Mit dieser Plattform soll der Lehrer bei den Aufgaben eine Schritt für Schritt Anleitung erstellen können. Benötigen die Schüler Hilfe bei einer Aufgabe, können sie die Aufgabe selbständig lösen und bekommen mit der Hilfe eine genaue Anleitung.


\subsection{Lösungsansatz}
\subsubsection{Konzeption}
Für die Konzeption des eigenen Lösungsansatzes war es sehr nützlich, dass die HSR selber eine Moodle Seite betreibt. So konnte bereits während dem Studium viel Erfahrung mit einer Lernplattform gesammelt werden. \\

\noindent Als Luca Gubler die BMS besuchte, hatte er zudem ein Login bei sofatutor. Von da konnten auch einige Ideen gesammelt werden. 


\subsubsection{Zentrale Elemente}
\subsubsection*{Benutzer und Rechte}
Es gibt die drei Gruppen ''Admin'', ''Lehrer'' und ''Schüler'', in welche eine Person eingeteilt werden kann. Die Admins sind in der Lage, die Lernplattform zu verwalten. Sie können neue Schüler und Lehrer erfassen und diese einer Klasse zuweisen. 

\noindent Lehrer können den Content ihrer Klasse verwalten und entscheiden, welche Fächer für die Schüler zugänglich sind. Zudem hat der Lehrer Zugriff auf die Statistiken seiner Klasse.  \\

\noindent Die Schüler sind lediglich in der Lage, den freigegebenen Content ansehen und die dazu gehörigen Aufgaben und Quizze lösen. Ferner sollen sie auch in der Lage sein, ihre eingenen Statistiken einzusehen.

\subsubsection*{Lerninhalte}
Zur Zeit können nur die Betreiber der Plattform neue Lerninhalte verfassen oder bearbeiten. Zu einem späteren Zeitpunkt sollen auch Lehrpersonen dazu in der Lage sein. Für die Bacheloarbeit steht jedoch nur ein begrenzter Zeitraum zur Verfügung  und dieses Feature wird bereits sehr gut in Moodle umgesetzt. Man hätte hier also kein Feature mit grossem Mehrwert implementieren können. Aus diesem Grund entschied man sich dafür, nicht weiter auf dieses Feature einzugehen. 

\subsubsection*{Übungen}
Lehrpersonen können neue Übungen erstellen. Zu diesem Zweck steht eine Website bereit, auf welcher die Aufgabe erfasst werden kann. Pro Übung können mehrere Aufgaben erstellt werden, welche jeweils mehrere Hilfestellungen enthalten können.







\newpage
\section{Technischer Bericht}
\subsection{Einleitung}

\subsubsection{Hintergrund}
Luca Gubler arbeitete neben dem Studium als IT Supporter in der Oberstufe Gossau ZH. Während eines Gespräches mit seinem Vorgesetzten bemängelte dieser, dass es keine guten E-Learning Plattformen für Schulen gibt. Zwar gebe es vereinzelt Lehrer oder Schulen, welche dies selber in die Hand nehmen und eine Open Source Tool wie Moodle verwenden und darauf den Content selber erfassen. Es wird jedoch viel Zeit benötigt, bis der gesamte Schulstoff digital erfasst wird. Solche Projekte würden auch oft nur halbherzig umgesetzt und geraten dann in Vergessenheit. Meistens lässt auch die Qualität zu wünschen übrig.

\subsubsection{Problemstellung / Vision}
Aus diesem Gespräch heraus entstand die Idee, eine All-In-One Lernplattform für Schulen zu erstellen. In einem ersten Schritt richtet sich die Plattform an die Sekundarstufe 1 und soll den ganzen Schulstoff in Form von Videos und Theorie-Zusammenfassungen zur Verfügung stellen. Zudem gibt es Übungen zu den einzelnen Themen. Sobald die Schüler eine Aufgabe gelöst haben, kann der Lehrer Statistiken einsehen, wie gut die Schüler diese Aufgaben gelöst haben. Falls zum Beispiel eine Aufgabe besonders schlecht gelöst wurde, kann der Lehrer diese Aufgabe mit der ganzen Klasse besprechen. Zudem kann der Lehrer auf einzelne Schüler zugehen, falls er bemerkt, dass diese Hilfe in einem spezifischen Aufgabengebiet benötigen. \\

Des Weiteren soll ein Forum bereitstehen, in welchem Schüler Fragen zu spezifischen Aufgaben stellen können. Falls der Lehrer sieht, dass auch hier die Schüler etwas nicht ganz verstanden haben, kann er nochmals direkt auf ein Thema eingehen. \\

Eine solche Plattform soll es den Schülern erlauben, unabhängig voneinander zu lernen. Falls ein Schüler ein Thema schneller lernt als andere, so soll er während der Schulzeit ein anderes Thema lernen können und nicht von anderen Schülern aufgehaltet werden. Trotzdem soll der Lehrer über den Wissensstand seiner Schüler informiert sein und diese gezielt unterstützen können.

\subsubsection{Aufgabenstellung}
Bei dieser Problemstellung handelt es sich um eine grobe Beschreibung der gesamten Idee. Da der Umfang dieser Bachelorarbeit begrenzt ist, können nicht alle Punkte umgesetzt werden. Die Idee für eine solche Idee ist jedoch nicht ganz neu und es gibt Ähnlichkeiten zu bereits existierenden Tools. Aus diesem Grund wurde nach einem Punkt gesucht, in welchem man sich von den existierenden Tools abheben kann. \\

In Zusammenarbeit mit Frank Koch, dem Betreuer dieser Bachelorarbeit und dem Moodle Experten der \gls{hsr}, kam man zum Schluss, dass man sich mit dem aufgabenspezifichen Teil der Anwendung von solchen Tools abgrenzen kann. Die Schüler sollen in der Lage sein, Aufgaben zu lösen. Falls diese jedoch nicht wissen, wie man beim einem Aufgabentyp vorgehen muss, können sie mit Hilfestellungen Schritt für Schritt durch die Aufgabe geleitet werden. So können neue wie auch schwierige und komplizierte Aufgaben trotzdem selbstständig erarbeitet werden. \\
%Schlussendlich soll jedoch eine lauffähige Anwendung existieren, also kommt einiges an Funktionalität hinzu. So muss zum Beispiel ein User Management existieren. Ein Administrator muss neue Lehrer und Schüler erfassen und diese einer Schulklasse zuweisen können.
%\\
%Auf der Lernplattform ist der gesamte Schulstoff der Sekundarstufe vorhanden. Die Lehrer können einzelne Fächer freischalten, so dass diese für die Schüler sichtbar sind. Haben die Schüler eine Aufgabe gelöst, kann der Lehrer sehen, wie viele Schüler diese Aufgabe gelöst haben und wie gut der Klassenschnitt ist.
%\\
%Die Schüler haben die Möglichkeit, die freigeschaltenen Fächer anzusehen und Quizze und Übungen zu lösen.

\subsubsection{Zielgruppen}
In einem ersten Schritt richtet sich der Aufgaben-Coach an alle drei Stufen der Sekundarstufe 1. Später soll der Aufgaben-Coach aber auch für die Primarstufe oder Sekundarstufe 2 zugänglich gemacht werden. 

\subsection{Stand der Technik}
Es gibt bereits einige Tools, welche Lerninhalte für Schüler zur Verfügung stellen. Diese bestehenden Tools lassen sich in zwei Kategorien einteilen:
\begin{itemize}
	\item Tools von kommerziellen Anbietern
	\item Open-Source Tools
\end{itemize}

\subsubsection{Kommerzielle Anbieter}
Sofatutor gehört wohl zu den grössten und bekanntesten Anbietern von Lerninhalten im \gls{dach} Raum. Sofatutor richtet sich jedoch hauptsächlich an Schüler, welche Nachhilfe in einem bestimmten Fach brauchen. In einem begrenzten Rahmen ist es auch möglich, sofatutor im Schulaltag einzubinden\footcite{sofatutor_fuer_lehrer}. Lehrer können ihren Schülern einzelne Videos oder Übungen freischalten. \\

EF Class ist ein weiterer Anbieter von Lerninhalten. Hier werden aber nur Inhalte rund um den English Unterricht angeboten. Um alle Fächer abzudecken, müsste eine Schule gleich mehrere solcher Tools einsetzen. \\

%TODO eigenes kapitel mit diesem abschnitt
%Mit dem Aufgaben-Coach verfolgt man aber das Ziel des ''Flipped Classroom''. Bei der klassischen Methode, wie sie zur Zeit in der Schule angewendet wird, lernen die Schüler die Theorie in der Schule und vertiefen das Wissen durch Übungen zu Hause. Beim ''flipped classroom'' stellt der Lehrer den Schülern die Theorie zum Beispiel als Video zur Verfügung. Die Schüler können die Theorie so zu Hause lernen und in der Schule dann die darauf aufbauenden Übungen lösen. So benötigen die Schüler nur dann die Hilfe des Lehrers, wenn sie vor einem Problem stehen. Der Lehrer steht also nur noch als eine Art Coach zur Verfügung. 

%TODO schreib über CMS / LMS und andere Tools
\subsubsection{Open Source}
Grundlegend fallen Open Source Lösungen in die Kategorie der \gls{lms}. \gls{lms} selber ist eine Unterkategorie von \gls{cms}. \gls{cms} werden typischerweise als \gls{ecm} oder \gls{wcm} verwendet. Ein \gls{ecm} ist auf die Arbeitswelt ausgerichtet und stellt Funktionen zur Dokumentverwaltung oder der Verwaltung digitaler Inhalte zur Verfügung. Zudem wird ein rollenbasierter Zugriff auf die Inhalte von Firmen ermöglicht. SharePoint\footcite{sharepoint} ist ein bekanntes \gls{ecm} und ermöglicht es Dateien zu speichern oder zu teilen, ist jedoch hauptsächlich für den firmeninternen Gebrauch ausgelegt. \\

Mit einem \gls{wcm} kann praktisch ohne grosses Vorwissen eine Website erstellt werden. So können zum Beispiel auf sehr einfache Weise neue Seiten erstellt und verwaltet werden. Gemäss dieser Statistik von Internet Live Stats\footcite{internet_live_stats} gibt es über 1.7 Milliarden unique Websites. Unter ''unique'' wird eine Website mit einzigartigem Hostname verstanden. Diese hohe Nummer ist aber haupstächlich wegen dem Gebrauch unterschiedlicher \gls{cms} so hoch, da auch laien sehr schnell Websites erstellen können. WordPress ist mit Abstand meist eingesetzte \gls{cms}. Mit über 27 Millionen Websites besitzt WordPress einem Marktanteil von 53.3\%. Der zweite Platz belegt Joomla! mit gerade einmal 3.8 Millionen Websites und einem Marktanteil von 7.5\%\footcite{cms_market_share}. Die Differenz dieser beiden \gls{cms} ist relativ hoch. \\

Um E-Learning Websites zu erstellen, gibt es nochmals eine Unterkategorie von \gls{wcm}, sogenannte \gls{lms}. Die Hauptfunktion solcher Tools ist das Bereitstellen von Online Trainings. Ein grosser Vorteil solcher Tools ist, dass sie sehr flexibel eingesetzt werden können. Eine Firma kann zum Beispiel interne Weiterbildungs Videos veröffentlichen, während eine Schule ihre Fächer über eine solche Plattform verwaltet. \\

Eines der wohl bekanntesten \gls{lms} ist Moodle\footcite{moodle_homepage}, welches auch an der \gls{hsr} zum Einsatz kommt. Moodle belegt jedoch nur Platz 19 der besten \gls{lms} basierend auf User Experience\footcite{moodle_ux}. Es ist jedoch anzumerken, dass diese Umfrage aus dem Jahr 2018 kommt. In der Umfrage von 2019 wurde Moodle nicht mehr aufgelistet. Platz 1 wird in beiden Jahren von Looop\footcite{looop_homepage} belegt. \\

Mit praktisch jedem \gls{lms} können Kurse den Teilnehmern zur Verfügung gestellt werden. Die meisten dieser Systeme richten sich aber an Firmen, welche Trainings für ihre Mitarbeiter bereitstellen möchten. \\

An diesem Punkt möchte man sich mit Aufgaben-Coaching abgrenzen. Man möchte eine Anwendung erstellen, welche spezifisch auf Schulen zugeschnitten ist.

%Unter einem \gls{cms} versteht man ein System, welches die die Erstellung und Verwaltung von digitalen Inhalten ermöglicht. 
%
%In diesem Kontext werden Open-Source Lösungen als \gls{lms} bezeichnet. Unter \gls{lms} versteht man eine Software, welches die Bereitstellung von Lerninhalten ermöglicht\footcite{learning_management_system}.
%Moodle ist die wahrscheinlich bekannteste Open Source LMS. Bei Moodle handelt es sich aber nur um die Plattform an sich. Im Gegensatz zu kommerziellen Lösungen wird hier kein Content zur Verfügung gestellt. Neben dem Content braucht es aber auch noch einen Verantwortlichen an der Schule, welcher sich um die Verwaltung von Moodle kümmert. \\
%
%Auf den ersten Blick hat der Aufgaben-Coach sehr viele Ähnlichkeiten zu Moodle. In Moodle ist es zwar möglich, Aufgaben zu erstellen und auszuwerten, eine enge Betreuung der Schüler ist jedoch nicht direkt möglich. \\
%
%An diesem Punkt möchte man mit dem Aufgaben-Coach ansetzen. Mit dieser Plattform soll der Lehrer bei den Aufgaben eine Schritt für Schritt Anleitung erstellen können. Benötigen die Schüler Hilfe bei einer Aufgabe, können sie die Aufgabe selbständig lösen und bekommen mit der Hilfe eine genaue Anleitung.


\subsection{Lösungsansatz}
\subsubsection{Konzeption}
Um ein Konzept für die eigene Lösung zu finden, war es sehr nützlich, dass die \gls{hsr} selber eine Moodle Seite betreibt. So konnte bereits während dem Studium Erfahrung mit einer Lernplattform gesammelt werden. \\

Als Luca Gubler die BMS besuchte, hatte er zudem ein Login bei sofatutor. Von da konnten auch einige Ideen zusammengetragen werden. 


\subsubsection{Zentrale Elemente}
\subsubsection*{Benutzer und Rechte}
Es gibt die Gruppen ''Administrator'', ''Lehrer'' und ''Schüler'', in welche eine Person eingeteilt werden kann. \\

Lehrer können den Inhalt ihrer Klasse verwalten und entscheiden, welche Fächer für die Schüler zugänglich sind. Sie erstellen Aufgaben und Quizzes, welche sie über einen Wochenplan einer Klasse zuweisen können. Zudem hat der Lehrer Zugriff auf verschiedene Statistiken über seine Klassen. \\

Schüler sind lediglich in der Lage, den freigegebenen Inhalt zu sehen und die dazu gehörigen Aufgaben und Quizze zu lösen. Ferner sollen sie auch in der Lage sein, Statistiken über ihren Lernfortschritt einzusehen.

\subsubsection*{Lerninhalte}
Zur Zeit können nur die Betreiber der Plattform neue Lerninhalte verfassen oder bearbeiten. Zu einem späteren Zeitpunkt sollen auch Lehrpersonen dazu in der Lage sein. Für die Bachelorarbeit steht jedoch nur ein begrenzter Zeitraum zur Verfügung und diese Funktionalität wird bereits sehr gut in Moodle umgesetzt. Man hätte hier also keinen Mehrwert generieren können. Aus diesem Grund entschied man sich dafür, nicht weiter darauf einzugehen. 

\subsubsection*{Aufgaben}
Lehrpersonen können neue Aufgaben erstellen. Auf der dafür erstellten Seite der Webappliaktion können diese erfasst werden. Pro Aufgabe können mehrere Fragen, oder auch ''Übungen'' erstellt werden. Eine solche Frage im Kontext Mathematik könnte zum Beispiel folgendes sein: ''Was ergibt 3/4 + 1/2?''. Für jede Frage können mehere Hilfestellungen erfasst werden. Eine Hilfestellung sollte dem Schüler die Möglichkeit bieten, Schritt für Schritt durch eine Frage geleitet zu werden. Der Sinn der Aufgaben ist es, den Schülern eine Möglichkeit zu bieten, den gelernten Stoff durch praktische Übung zu vertiefen und die Lehrperson auf dem aktuellen Stand zu halten.

%\subsubsection*{Quizzes}
%Auf der dafür erstellten Seite der Webapplikation, können Lehrpersonen neue Quizzes erstellen. Ein Quiz besteht wie eine Aufgabe aus mehreren verschiedenen Fragen. Für jede Frage gibt es eine vielzahl von Antwortmöglichkeiten, von denen eine bis mehrere richtig sind. Mit Hilfe der automatisierten Auswertung der Quizze soll die Lehrperson einen Überblick bekommen, wie gut eine Klasse, aber auch einzelne Schüler, ein bestimmtes Thema verstanden haben. Der Vorteil von Quizzes gegenüber klassischen Prüfungen ist, dass der Lehrer bereits während dem Unterrichten eines Themas sieht, wie der aktuelle Stand der Klasse ist. Quizze können Lehrpersonen also direkt bei der Gestalltung des Unterrichts unterstützen.



\newpage
\section{Technischer Bericht}
\subsection{Einleitung}

\subsubsection{Hintergrund}
Luca Gubler arbeitete nebst dem Studium als IT Supporter in der Schule Gossau ZH. Während eines Gespräches mit seinem Vorgesetzten bemängelte dieser, dass es keine guten E-Learning Plattformen für Schulen gibt. Zwar gebe es vereinzelt Lehrer oder Schulen, welche dies selber in die Hand nehmen und ein Open Source Tool wie Moodle verwenden, jedoch wird viel Zeit benötigt, bis der gesamte Schulstoff digital erfasst wurde. Solche Projekte werden oft auch nur halbherzig umgesetzt und geraten schnell in Vergessenheit. Meistens lässt auch die Qualität zu wünschen übrig.

\subsubsection{Problemstellung / Vision}
Aus diesem Gespräch heraus entstand die Idee, eine All-In-One Lernplattform für Schulen zu entwickeln. In einem ersten Schritt richtet sich die Plattform an die Sekundarstufe 1 und soll den ganzen Schulstoff in Form von Videos und Theorie-Zusammenfassungen zur Verfügung stellen. Diese Anwendung soll es Lernenden ermöglichen, sich ein neues Thema komplett selbstständig beizubringen. Um die erlernte Theorie mit praktischen Beispielen zu vertiefen, sollen Übungen zur Verfügung stehen. Den Lernenden soll es auch möglich sein, diese Plattform zu Hause zu verwenden. \\

Dies erlaubt es einer Lehrperson, die sogenannte Flipped Classroom Unterrichtsmethode anzuwenden. Mit dieser Methode erlernen Lernende die Theorie zu einem Thema selbstständig, was üblicherweise in Form von Hausaufgaben stattfindet. In der Schule werden dann Übungen oder Projektarbeiten gelöst. Die Lehrperson ist nicht mehr im klassischen Sinne als Wissensvermittler zuständig, sondern sie nimmt die Rolle eines Coaches ein. Lernende, welche Hilfe benötigen, können individuell unterstützt werden. Die anderen Lernenden werden dadurch aber nicht abgelenkt und können an einem anderen Thema weiter arbeiten. \\

Da die Lernenden aber weitgehend selbständig arbeiten, kann die Lehrperson schnell den Überblick über den Stand der einzelnen Lernenden verlieren. Ein zentraler Punkt in der Schule sind die Prüfungen. Da die Lehrperson auf keine bösen Überraschen stossen soll, sollen Statisiken zur Verfügung stehen, mit welchen der aktuelle Wissensstand der Lernenden fortlaufend überprüft werden kann. Dies erlaubt es einer Lehrperson, die Lernenden da zu unterstützen, wo sie am meisten Hilfe benötigen. So besteht auch die Möglichkeit, die gesamte Klasse auf ein Niveau zu heben, auf welchem der Wissensstand effektiv geprüft werden kann. Die Lehrperson hat einen Überblick über die Lernenden, auch wenn er nicht genaustens vorgibt, an was diese wann arbeiten sollen. \\

Mit dieser Lernplattform hat man sich das Ziel gesetzt, Lehrperson und Lernende enger miteinander zu verbinden. Um diese Zusammenarbeit weiter zu fördern, soll ein aufgabenspezifisches Forum bereit gestellt werden. In diesem können Lernende Fragen zu einzelnen Aufgaben stellen und sich so gegenseitig helfen. Die Lehrperson muss nur dann eingreifen, wenn sich die Lernenden etwas falsch erklären oder niemand reagiert.

\subsubsection{Aufgabenstellung}
Bei dieser Problemstellung handelt es sich um eine grobe Beschreibung der gesamten Idee. Da die Zeit dieser Bachelorarbeit begrenzt ist, können nicht alle Punkte in vollem Umfang umgesetzt werden. Die Idee für eine solche Applikation ist jedoch nicht ganz neu und es gibt Ähnlichkeiten zu bereits existierenden Tools. Aus diesem Grund wurde nach einem Bereich gesucht, in welchem man sich von den existierenden Tools abheben kann. \\

In Zusammenarbeit mit Frank Koch, dem Betreuer dieser Bachelorarbeit und dem Moodle Experten der \gls{hsr}, kam man zum Schluss, dass man sich mit dem aufgabenspezifichen Teil der Anwendung, von solchen Tools abgrenzen kann. Lernende sollen in der Lage sein, sich ein Thema selbstständig beizubringen. Zu der bereitgestellten Theorie sollen Übungen bereit stehen, mit welchen die Theorie vertieft werden kann. Damit die Lernenden auch schwierige oder neue Aufgaben selbstständig erarbeiten können, soll eine Schritt für Schritt Anleitung zur Verfügung stehen. So können neue wie auch schwierige und komplizierte Aufgaben selbstständig erarbeitet werden.

\subsubsection{Zielgruppen}
Für den ersten Moment ist die Applikation Aufgaben-Coach für die Sekundarstufe 1 ausgelegt. Später soll es jedoch auch möglich sein, den Aufgaben-Coach in anderen Schulstufen einzusetzen.

\subsection{Stand der Technik}
\subsubsection{Existierende Produkte}
Es gibt bereits einige Tools, welche Lerninhalte für Schüler zur Verfügung stellen. Diese lassen sich grundsätzlich in die beiden Kategorien ''kommerzielle Anbieter'' und ''Open Source Projekte'' einteilen.

\subsubsection*{Kommerzielle Anbieter}
Sofatutor gehört wohl zu den grössten und bekanntesten Anbietern von Lerninhalten in \gls{dach}. Sofatutor richtet sich jedoch hauptsächlich an Schüler, welche Nachhilfe in einem bestimmten Fach brauchen. In einem begrenzten Rahmen ist es auch möglich, Sofatutor in den Schulaltag einzubinden\footcite{sofatutor_fuer_lehrer}. Lehrer können ihren Schülern zum Beispiel einzelne Videos oder Übungen freischalten. \\

EF Class ist ein weiterer Anbieter von Lerninhalten. Dieses Tool ist aber speziell für den Englisch Unterricht ausgelegt\footcite{ef_class_homepage}. Möchte eine Lehrperson die Flipped Classroom Methode anwenden, müssten gleich mehrere solcher Tools eingekauft werden. Zum einen ist das finanziell gesehen ein negativer Punkt und zum anderen ist es schwer einen Gesamtüberblick zu behalten, da die Informationen über den Wissensstand der Lernenden auf mehrere Plattformen verteilt ist.

%TODO eigenes kapitel mit diesem abschnitt
%Mit dem Aufgaben-Coach verfolgt man aber das Ziel des ''Flipped Classroom''. Bei der klassischen Methode, wie sie zur Zeit in der Schule angewendet wird, lernen die Schüler die Theorie in der Schule und vertiefen das Wissen durch Übungen zu Hause. Beim ''flipped classroom'' stellt der Lehrer den Schülern die Theorie zum Beispiel als Video zur Verfügung. Die Schüler können die Theorie so zu Hause lernen und in der Schule dann die darauf aufbauenden Übungen lösen. So benötigen die Schüler nur dann die Hilfe des Lehrers, wenn sie vor einem Problem stehen. Der Lehrer steht also nur noch als eine Art Coach zur Verfügung. 

%TODO schreib über CMS / LMS und andere Tools
\subsubsection*{Open Source Projekte}
Open Source Anwendungen, wie zum Beispiel Moodle, fallen in die Kategorie der \gls{lms}. \gls{lms} ist eine Unterkategorie von \gls{cms}. Diese werden typischerweise als \gls{ecm} oder \gls{wcm} verwendet. Ein \gls{ecm} ist auf die Arbeitswelt ausgerichtet und stellt Funktionen zur Dokumentenverwaltung oder der Verwaltung digitaler Inhalte zur Verfügung. Zudem wird ein rollenbasierter Zugriff auf die Inhalte ermöglicht. SharePoint\footcite{sharepoint} ist ein bekanntes \gls{ecm} und ermöglicht es Dateien zu speichern oder zu teilen, ist jedoch, wie man in \gls{ecm} heraushören kann, hauptsächlich für den firmeninternen Gebrauch ausgelegt. \\

Mit einem \gls{wcm} kann praktisch ohne grosses Vorwissen eine Webseite erstellt werden. So können zum Beispiel auf eine sehr einfache Art und Weise neue Seiten erstellt und verwaltet werden. Gemäss der Statistik von Internet Live Stats\footcite{internet_live_stats} gibt es über 1.7 Milliarden unique Websites. Unter ''unique Websites'' wird eine Webseite mit einzigartigem Hostnamen verstanden. Diese Anzahl ist aber haupstächlich wegen dem Gebrauch unterschiedlicher \gls{cms} so hoch, da auch Laien sehr schnell Webseiten erstellen können. Unter allen \gls{cms} ist WordPress das mit Abstand meist eingesetzte Tool und besitzt mit über 27 Millionen Webseiten einem Marktanteil von 53.3\%. Der zweite Platz belegt Joomla! mit gerade einmal 3.8 Millionen Webseiten und einem Marktanteil von 7.5\%\footcite{cms_market_share}.\\

Die Hauptfunktion solcher \gls{lms} Tools ist das Bereitstellen von Online Trainings. Ein grosser Vorteil solcher Tools ist, dass sie sehr flexibel eingesetzt werden können. Eine Firma kann zum Beispiel interne Weiterbildungen anbieten, während eine Schule den Schülern die Theorie zu einzelnen Themen anbietet. \\

Eines der wohl bekanntesten \gls{lms} ist Moodle\footcite{moodle_homepage}, welches auch an der \gls{hsr} zum Einsatz kommt. Moodle belegt jedoch nur Platz 19 von 20 der besten \gls{lms} Systemen, basierend auf User Experience\footcite{moodle_ux}. Diese Umfrage stammt jedoch aus dem Jahr 2018, in der Umfrage von 2019 wird Moodle gar nicht mehr aufgelistet. Moodle schnitt mit einer Bewertung von 70\% ab, während Looop\footcite{looop_homepage} den 1. Platz mit einer Wertung von 90\% besetzte. Es ist jedoch nicht ersichtlich, warum Moodle schlecht abschnitt, denn 88\% der Bentuzer würden Moodle weiterempfehlen.


%Unter einem \gls{cms} versteht man ein System, welches die die Erstellung und Verwaltung von digitalen Inhalten ermöglicht. 
%
%In diesem Kontext werden Open-Source Lösungen als \gls{lms} bezeichnet. Unter \gls{lms} versteht man eine Software, welches die Bereitstellung von Lerninhalten ermöglicht\footcite{learning_management_system}.
%Moodle ist die wahrscheinlich bekannteste Open Source LMS. Bei Moodle handelt es sich aber nur um die Plattform an sich. Im Gegensatz zu kommerziellen Lösungen wird hier kein Content zur Verfügung gestellt. Neben dem Content braucht es aber auch noch einen Verantwortlichen an der Schule, welcher sich um die Verwaltung von Moodle kümmert. \\
%
%Auf den ersten Blick hat der Aufgaben-Coach sehr viele Ähnlichkeiten zu Moodle. In Moodle ist es zwar möglich, Aufgaben zu erstellen und auszuwerten, eine enge Betreuung der Schüler ist jedoch nicht direkt möglich. \\
%
%An diesem Punkt möchte man mit dem Aufgaben-Coach ansetzen. Mit dieser Plattform soll der Lehrer bei den Aufgaben eine Schritt für Schritt Anleitung erstellen können. Benötigen die Schüler Hilfe bei einer Aufgabe, können sie die Aufgabe selbständig lösen und bekommen mit der Hilfe eine genaue Anleitung.

\subsubsection{Flipped Classroom}
Plattformen wie Moodle erlauben es einer Lehrperson, das Flipped Classroom Prinzip in den Unterricht zu integrieren. In der klassischen Unterrichtsmethode besteht der Unterricht aus einer Phase, in welcher die Lehrperson den Schülern die Theorie zu einem neuen Thema beibringt. In der zweiten Phase lösen die Schüler basierend auf der erlernten Theorie Übungen. Diese finden dann meistens zu Hause in Form von Hausaufgaben statt. Flipped Classroom vertauscht nun diese beiden Phasen. Die Lerninhalte werden zu Hause von den Schülern erarbeitet. Dies kann in Form von Videos oder Theoriezusammenfassungen stattfinden. Die Anwendung, also das Lösen von Aufgaben, findet dann während dem Unterricht statt\footcite{flipped_classroom_theorie}. \\

Da die Schüler die Theorie zu Hause lernen, hat eine Lehrperson während des Unterrichts viel mehr Zeit, sich direkt und individuell mit den Schülern zu befassen. Die Lehrperson kann jene Schüler coachen, welche ein Thema noch nicht ganz verstehen. Die Schüler, welche das Thema aber verstehen, können bereits mit dem nächsten Thema beginnen und werden nicht aufgehalten. \\

Auf den ersten Blick bietet Flipped Classroom viele Vorteile. Nachfolgend befindet sich eine kurze Übersicht mit Vor- und Nachteilen\footcite{flipped_classroom_pro_con}.

\newpage
\subsubsection*{Vorteile}
\begin{itemize}
	\item Die Schüler haben mehr Kontrolle über ihr eigenes Lernen. Sie können selber entscheiden, in welchem Tempo sie etwas lernen. Fragen oder Unklarheiten können dann im Unterricht mit einer Lehrperson besprochen werden. Jene Schüler, welche ein Konzept nicht auf Anhieb verstehen, können sich die Theorie in Ruhe nochmals anschauen, ohne den Anschluss an den Unterricht zu verlieren.
	\item Während des Unterrichts gibt es weniger Frontalunterricht, dafür mehr Übungen und Projektarbeiten. Die Schüler können sich so gegenseitig unterstützen und beim lernen helfen. Die Lehrperson ist nur noch als Coach tätig und unterstützt einzelne Schüler.
	\item Die Lektionen sind immer verfügbar. Kann ein Schüler wegen Krankheit nicht in den Unterricht kommen, so kann die Theorie dennoch erlernt werden. 
	\item Weil die Lektionen praktisch 24/7 verfügbar sind, können die Eltern auch zu Hause schauen, was genau die Kinder im Unterricht machen. So können sie sich auch selbst besser vorbereiten, falls sie ihre Kinder unterstützen wollen. 
\end{itemize}

\subsubsection*{Nachteile}
\begin{itemize}
	\item Ein grosser Nachteil ist, dass die Infrastruktur vorhanden sein muss. In der Schule werden den Schülern unter Umständen Tablets verteilt, auf welchen sie die Videos anschauen können. Zu Hause braucht der Schüler aber ein eigenes Tablet und einen Internet Anschluss. Es kann durchaus vorkommen, dass diese Infrastruktur nicht in jedem Haushalt vorhanden ist. \\
	In der Schule sollte dies jedoch kein Problem sein, da die Digitalisierung der Schulen immer weiter voran schreitet\footcite{digitale_schule}.
	\item Die Schüler werden dazu aufgefordert, sich zu Hause vorzubereiten. Es kann aber nicht garantiert werden, dass die Schüler dies auch tatsächlich tun.
	\item Bei Flipped Classroom verbringen die Schüler automatisch mehr Zeit vor einem Bildschirm. Es kann Eltern geben, welche es jedoch nicht toll finden, wenn ihre Kinder noch längere Zeit auf einen Bildschirm starren.
\end{itemize}

\subsection{Eigener Lösungsansatz}
\subsubsection{Konzeption}
Mit dem Aufgaben-Coach möchte man eine neue Anwendung erschaffen, welche sich von existierenden Anwendungen abgrenzt. Aufgaben-Coach richtet sich direkt an Schulen, welche so das Flipped Classroom Prinzip in den Unterricht integrieren wollen. \\

Diese Anwendung wurde mit den Bedürfnissen von Lehrenden und Lernenden entworfen. Für die Konzeption war es hilfreich, dass die \gls{hsr} selber eine Moodle Seite betreibt. So konnte bereits während dem Studium Erfahrung mit einem \gls{lms} gesammelt werden. Dies diente teilweise als Inspiration für den Entwurf der Mockups.

\subsubsection{Zentrale Elemente}
\subsubsection*{Benutzer und Rechte}
Eine Person kann einer der drei Rollen ''Administrator'', ''Lehrer'' oder ''Schüler'' zugewiesen werden. Aus Erfahrung weiss man, dass Schüler sehr experimentierfreudig sind. Aus diesem Grund wurde bei der Implementation besonders darauf geachtet, dass die Zugriffe auf die einzelnen Seiten geregelt ist. Schülern soll es zum Beispiel nicht möglich sein, Aufgaben anzupassen oder Statistiken anzusehen, auch wenn die korrekte \gls{url} im Browser eingegeben wird. \\

Der Administrator ist für das Erfassen der einzelnen Personen und der Zuweisung in die Klassen zuständig. \\

Lehrer können den Inhalt ihrer Klasse verwalten und entscheiden, welche Fächer für die Schüler zugänglich sind. Sie erstellen Aufgaben, welche sie über einen Wochenplan einer Klasse zuweisen können. Zudem hat der Lehrer Zugriff auf verschiedene Statistiken zu seinen Klassen. \\

Schüler sind in der Lage, den freigegebenen Inhalt zu sehen und die dazugehörigen Aufgaben zu lösen. Zusätzlich haben sie auch Zugriff auf ein Forum, in welchem Fragen zu Aufgaben gestellt werden können.

\subsubsection*{Lerninhalte}
Die Lehrer sind in der Lage, einzelne Fächer für die Schüler einer Klasse freizuschalten. Im Moment gibt es keine Möglichkeit, dass Lehrer selber Anpassungen an Inhalten vornehmen können. Nur die Betreiber von Aufgaben-Coach sind in der Lage, neue Theorieinhalte zu erstellen oder zu bearbeiten.

\subsubsection*{Aufgaben}
Lehrpersonen können neue Aufgaben erstellen, welche anschliessend von den Schülern gelöst werden können. Pro Aufgabe können mehrere Fragen erstellt werden. Für jede dieser Fragen können Hilfestellungen erfasst werden. Diese Hilfestellungen sollen es dem Schüler ermöglichen, auch schwierige Fragen selber beantworten zu können. Ein Schüler hat so die Möglichkeit, die erlernte Theorie mit praktischen Aufgaben zu vertiefen. \\

Wurde eine Aufgabe abgegeben, kann diese von einer Lehrperson korrigiert werden. Dazu kann mit Filteroptionen nach abgegebenen Aufgaben gesucht werden. Wählt die Lehrperson die abgegebene Aufgabe eines Schülers aus, kann er für jede einzelne Frage sagen, ob diese richtig oder falsch beantwortet wurde und dafür Punkte vergeben. Sollte die Aufgabe jedoch falsch gelöst worden sein, kann diese mit einem Feedback an den Schüler zurückgewiesen werden. Dieser hat dann nochmals die Möglichkeit, die Aufgabe zu verbessern. 

\subsubsection*{Statistiken}
Mit allen abgegebenen Aufgaben wird eine Statistik erstellt. Anhand dieser Statistiken kann eine Lehrperson erkennen, wie gut die Klasse oder ein einzelner Schüler ein Thema versteht. Dies erlaubt der Lehrperson auch, gezielt auf diese Schüler einzugehen und die Wissenslücken zu schliessen.

\subsubsection*{Forum}
Einem Schüler sollte es mit Hilfe der Hilfestellungen möglich sein, eine Aufgabe selbstständig zu lösen. Sollten dennoch Fragen oder Unklarheiten auftreten, kann er diese in einem Forum stellen, welches sich rund um diese einzelne Aufgabe dreht. In diesem Forum befinden sich auch bereits früher gestellte Fragen, welche die Frage des Schülers vielleicht gleich beantwortet.

%\subsubsection*{Quizzes}
%Auf der dafür erstellten Seite der Webapplikation, können Lehrpersonen neue Quizzes erstellen. Ein Quiz besteht wie eine Aufgabe aus mehreren verschiedenen Fragen. Für jede Frage gibt es eine vielzahl von Antwortmöglichkeiten, von denen eine bis mehrere richtig sind. Mit Hilfe der automatisierten Auswertung der Quizze soll die Lehrperson einen Überblick bekommen, wie gut eine Klasse, aber auch einzelne Schüler, ein bestimmtes Thema verstanden haben. Der Vorteil von Quizzes gegenüber klassischen Prüfungen ist, dass der Lehrer bereits während dem Unterrichten eines Themas sieht, wie der aktuelle Stand der Klasse ist. Quizze können Lehrpersonen also direkt bei der Gestalltung des Unterrichts unterstützen.



\newpage
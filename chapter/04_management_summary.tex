\section{Management Summary}
%\addcontentsline{toc}{section}{\protect\numberline{}Management Summary}

\subsection{Ausgangslage}
Mit der Umsetzung des Lehrplan21 gibt es in allen 21 deutsch- und mehrsprachigen Kantonen den selben Lehrplan. Unter den Zielen des Lehrplan21 findet man folgenden Punkt\footcite{lp21_ziel}:

\begin{displayquote}
Ein gemeinsamer Lehrplan ist eine Grundlage für die Koordination der Lehrmittel und erleichtert die gemeinsame Entwicklung von Lehrmitteln für die deutschsprachige Schweiz.
\end{displayquote}

Das heutige Unterrichtsmodell hat verschiedene Nachteile. Kann zum Beispiel ein Lernender aufgrund von Krankheit nicht in die Schule, so wird der gesamte Schulstoff verpasst. Diese Person ist jedoch dazu gezwungen, den verpassten Stoff nachzuholen. Es bleibt nichts anderes übrig, als sich zu Hause den verpassten Stoff selber beizubringen. Da ein Grossteil des Unterrichts daraus besteht, dass eine Lehrperson den Schülern die Theorie beibringt, kann die Lehrperson nicht individuell auf die Bedürfnisse der Schüler eingehen. Manche haben die Theorie bereits verstanden, während andere längst den Faden verloren haben. \\

Ein weiteres Problem der Lehrpersonen ist, dass diese erst sehr spät ein Feedback über den Wissensstand der Schüler bekommen. Erst wenn eine Prüfung durchgeführt wurde, sieht der Lehrer, wie gut die einzelnen Schüler der Klasse ein Thema verstanden haben. Bemerkt eine Lehrperson zu diesem Zeitpunkt, dass eine Wissenslücke besteht, ist meist nicht mehr genügend Zeit vorhanden, um dieses Thema nochmals in Ruhe zu erklären. \\


Die Idee von Lernplattformen für Schulen ist nicht ganz neu. An der \gls{hsr} wird zum Beispiel Moodle\footcite{moodle_homepage} eingesetzt, ein Open-Source \gls{lms}. Es gibt aber auch komerzielle Lösungen wie sofatutor\footcite{sofatutor_homepage} oder EF Class\footcite{ef_class_homepage}. Diese richten sich aber entweder an die Schüler direkt oder sind nur für einzelne Fächer konzipiert. \\

Mit dieser Bachelorarbeit möchte man eine Anwendung schaffen, welche direkt im Unterricht verwendet werden kann. Lernende sollen in der Lage sein, unabhängig und individuell einzelne Themen zu erlernen, während eine Lehrperson dennoch in der Lage ist, denn aktuellen Wissensstand einzelner Lernenden zu überprüfen. So ist die Lehrperson kein Wissensvermittler im klassischen Sinne, sondern er coachet die Lernenden und unterstützt diese beim Lernen.

% Bei dieser Bachelorarbeit wird eine Lernplattform erstellt, welche genau dieses Problem angeht. Auf dieser Lernplattform wird die Theorie der Schulfächer in Form von Theoriezusammenfassungen, Videos, Übungen und Quizze zur Verfügung gestellt. Schüler können mit Hilfe dieser Lernplattform unabhängig von ihrem Standort lernen, ob das nun in einer Freistunde während des Schulalltags oder zu Hause am Abdend im Bett ist. Alles, was die Schüler benötigen, ist ein Tablet und eine Internetverbindung. 

% Der Lehrer hat den Vorteil, dass er viel Zeit für die Vorbereitung der Stunden einsparen kann. Durch die Quizze und Übungen sieht der Lehrer auch direkt, auf welchem Stand seine Schüler sind und kann jene Schüler unterstützen, welche ein bestimmtes Thema noch nicht richtig verstanden haben. Schüler, welche aber alles ohne Probleme verstehen, können unanhängig von den anderen mit dem nächsten Thema fortfahren. 

\subsection{Vorgehen / Technologien}
Die Anwendung soll den Schülern rund um die Uhr zur Verfügung stehen und plattformübergreifend verwendbar sein. So kann die Anwendung auch zu Hause verwendet werden. Um diesen Anforderungen gerecht zu werden, wird eine Webanwendung mit dem Django-Framework und der Programmiersprache Python entwickelt. \\
Es wurde besonders darauf geachtet, dass die Anwendung modular aufgebaut ist. In Zukunft können auf der bestehenden Plattform weitere Features implementiert werden.

\subsection{Ergebnisse}
Während dieser Bachelorarbeit ist eine Anwendung mit dem Titel ''Aufgaben-Coaching'' entstanden. Diese Anwendung erlaubt es, Theoriezusammenfassungen mit Videos und Bildern bereitzustellen. Lehrer sind in der Lage, neue Aufgaben zu erstellen und diese den Schülern zur Verfügung zu stellen. Haben die Schüler die Aufgaben gelöst, kann der Lehrer diese einsehen und korrigieren. Anhand der gelösten Aufgaben werden Statistiken generiert, mit welchen eine Lehrperson jederzeit den Wissensstand der Klasse einsehen kann. Bemerkt die Lehrperson, dass ein einzelner Lernender Schwierigkeiten in einem spezifischen Aufgabengebiet hat, so kann die Lehrperson direkt und individuell auf diese Person zugehen, ohne die anderen Lernenden zu vernachlässigen.

\subsection{Ausblick}
Die Idee für dieses Bachelorarbeits-Thema stammt von Luca Gubler. Da für die Bachelorarbeit aber nur ein begrenzter Zeitraum zur Verfügung steht, konnten nicht alle Features im gewünschten Umfang umgesetzt werden. \\

Es ist jedoch geplant, dass diese Arbeit nach dem Studium in Form eines Startups weiter verfolgt wird.


\newpage
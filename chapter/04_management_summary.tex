\section*{Management Summary}
\addcontentsline{toc}{section}{\protect\numberline{}Management Summary}

\subsection*{Ausgangslage}
In den einzelnen Schulfächern der Sekundarstufe 1 wird seit geraumer Zeit immer das selbe unterrichtet. Der Lehrer steht vor der Klasse und bringt seinen Schülern ein neues Thema bei. Das Wissen, welches der Lehrer jedoch vermittelt, zählt zum Grundwissen und dieses hat sich in den letzten Jahren nicht Grundlegen verändert. Am Satz von Pythagoras ändert sich inhaltlich nichts. Somit erzählt der Lehrer Jahr für Jahr das selbe. Dies wiederum bedeutet, dass sich die Lehrer mehr auf den Schulstoff konzentriert und nicht auf das, was wirklich wichtig ist - die Schüler. \\

\noindent Die Schüler sitzen in der Schule und hören dem Lehrer zu. Schüler sind aber keine Maschinen. Manchmal sind sie krank oder unaufmerksam. Den verpassten Schulstoff müssen sie trotzdem irgendwie nachholen, spätestens wenn die Prüfung bevorsteht. Aber auch die Repetition ist mit viel Zeitaufwand verbunden. \\

\noindent Die Idee von Lernplattformen für Schulen ist nicht ganz neu. Auf den ersten Blick wirkt diese Idee sehr ähnlich zu Moodle, einem Open-Source Learning Management System, welches auch an der HSR eingesetzt wird. \\

\noindent Mit dem Lerncoach möchte man sich aber von Moodle abgrenzen, in dem er eine bessere Betreuung der Schüler ermöglicht. Wenn der Lehrer eine Aufgabe erstellt, kann er zudem eine Schritt für Schritt Anleitung erstellen. Falls der Schüler die Antwort nicht sofort weiss, wird er mit Hilfe dieser Anleitung durch die Aufgabe geführt und kommt so zum richtigen Resultat. 

% Bei dieser Bachelorarbeit wird eine Lernplattform erstellt, welche genau dieses Problem angeht. Auf dieser Lernplattform wird die Theorie der Schulfächer in Form von Theoriezusammenfassungen, Videos, Übungen und Quizze zur Verfügung gestellt. Schüler können mit Hilfe dieser Lernplattform unabhängig von ihrem Standort lernen, ob das nun in einer Freistunde während des Schulalltags oder zu Hause am Abdend im Bett ist. Alles, was die Schüler benötigen, ist ein Tablet und eine Internetverbindung. 

% Der Lehrer hat den Vorteil, dass er viel Zeit für die Vorbereitung der Stunden einsparen kann. Durch die Quizze und Übungen sieht der Lehrer auch direkt, auf welchem Stand seine Schüler sind und kann jene Schüler unterstützen, welche ein bestimmtes Thema noch nicht richtig verstanden haben. Schüler, welche aber alles ohne Probleme verstehen, können unanhängig von den anderen mit dem nächsten Thema fortfahren. 

\subsection*{Vorgehen / Technologien}
Da der Lerncoach Plattformübergreifend funktionieren soll, wurde eine Webanwendung mit Python und dem Webframework Django entwickelt. Die Anwendung wurde so konzipiert, das pro Schule eine separate Instanz zur Verfügung gestellt wird.

\subsection*{Ergebnisse}
Im Zuge dieser Bachelorarbeit ist grundlegende Lernplattform erstellt worden. Es können Theoriezusammenfassungen mit Videos und Bildern erstellt werden. Lehrer sind in der Lage, neue Aufgaben oder Quizze zu erstellen und den Schülern zur Verfügung zu stellen. Haben die Schüler die Aufgaben gelöst, kann der Lehrer diese einsehen und korrigieren. Falls die Lehrperson bemerkt, dass ein Schüler Mühe in einem spezifischen Aufgabengebiet hat, kann er direkt auf diese Schüler zugehen.

\subsection*{Ausblick}
Nach der Bachelorarbeit soll diese Arbeit weiter in Form eines Startups verfolgt werden. Um produktiv in einer Schule eingesetzt werden zu können, müssen jedoch noch einige Features verbessert werden. Ein zentraler Punkt dabei sind die Statistiken für den Lehrer. Des weiteren muss das User Interface noch weiter ausgearbeitet werden.


\newpage
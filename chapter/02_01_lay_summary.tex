\section*{Lay Summary}

Mit der, im Rahmen dieser Bachelorarbeit, entwickelten Webanwendung, soll die Zusammenarbeit von Schülern und Lehrern verbessert werden. Dabei spielt die Unterrichtsmethode ''Flipped Classroom'', also ''Umgedrehter Unterricht'', eine wichtige Rolle. Hierbei handelt es sich um eine Unterrichtsmethode, bei welcher die Schüler ihr Wissen zu einem grossen Teil selbstständig erarbeiten und sich die Rolle des Lehrers mehr und mehr in die eines Coaches verwandelt. Doch auch in der Rolle eines Coaches muss die Lehrperson über Möglichkeiten verfügen, den Wissensstand der Schüler zu ermitteln um sicherzustellen, dass niemand den Anschluss verliert.


Mit Hilfe der entwickelten Anwendung können Schüler selbstständig vordefinierte Theorieinhalte durcharbeiten und lernen. Um die Schüler beim Lernen so gut wie möglich zu unterstützen sowie zu überprüfen, ist es Lehrpersonen möglich, selber Aufgaben zu definieren, welche sie den Schülern als Hausaufgabe aufgeben können. Zu jeder Aufgabe können Hilfestellungen erfasst werden, welche den Schülern beim Lösen, helfen sollen. Sollte dies nicht genügen, kann im aufgabenspezifischen Forum um Hilfe gebeten werden. Die von den Schülern gelösten Aufgaben können von der Lehrperson korrigiert und bewertet werden. Anhand der bewerteten Abgaben kann die Lehrperson Statisiken einsehen, welche ihr den aktuellen Wissenstand der einzelnen Schüler übermittelt. Dadurch können schwächere Schüler erkannt und individuell unterstützt werden. 


\newpage
\newpage
\subsection{Ergebnisse}

\subsubsection{Resultate}
Trotz des verkürzten Zeitplans war es dennoch möglich, einen Grossteil der Funtkionen umzusetzen. Nachfolgend wird aufgelistet, welche Funktionen durch Aufgaben-Coaching ermöglicht werden. Die Liste ist auf die einzelnen User Rollen aufgeteilt.

\subsubsection*{Student}
Nachfolgend wird die Funktionalität der Schüler aufgelistet:

\begin{itemize}
	\item Einzelne Lektionen anschauen
	\item Aufgaben lösen, falls nötig auch verbessern
	\begin{itemize}
		\item Pro Frage können mehrere Hilfestellungen bereit stehen, welche dem Schüler beim Beantworten dieser Frage helfen.
	\end{itemize}
	\item Fragen im Forum stellen
\end{itemize}

\subsubsection*{Teacher}
Lehrer haben folgende Funktionalität:

\begin{itemize}
	\item Übungen erfassen und bearbeiten
	\begin{itemize}
		\item Dynamisches hinzufügen von Fragen und Hilfestellungen.
		\item Punktevergabe pro Frage.
	\end{itemize}
	\item Eigene Klassen verwalten
	\begin{itemize}
		\item Fächer einer Klasse zuweisen
		\item Aufgaben dem Wochenplan einer Klasse zuweisen
	\end{itemize}
	\item Gelöste Aufgaben korrigieren
	\begin{itemize}
		\item Filtern von allen gelösten Aufgaben seiner Schüler
		\item Pagination Ansicht der gefilterten Aufgaben
		\item Einzelne Aufgaben akzeptieren oder zurückweisen
		\item Feedback für einzelne Aufgaben erfassen
		\item Mail an Schüler mit dem Feedback schicken
	\end{itemize}
	\item Statistiken einsehen
	\begin{itemize}
		\item Von gelösten Aufgaben werden Statistiken erstellt
		\item Lehrer kann diese Statistiken einsehen und so schwächere Schüler besser unterstützen
	\end{itemize}
\end{itemize}


\subsubsection*{Admin}
Im Prinzip hat der Admin die Rechte, um alles auf der Website zu tun. Nachfolgend sind nur die Features, welche sich spezifisch an den Admin richten:

\begin{itemize}
	\item Schüler und Lehrer erfassen und bearbeiten
	\item Klassen erstellen und User diesen Klassen zuweisen
\end{itemize}


\subsubsection{Verbesserung}
Die Anwendung war viel komplexer als zu Beginn erwartet. Um im Zeitplan zu bleiben, musste bei gewissen Features  Kompromisse eingegangen werden, um dennoch im Zeitplan zu bleiben. Folgende Liste mit Features soll noch verbessert werden:

\begin{itemize}
	
\end{itemize}

\subsubsection{Weiterentwicklung}

\newpage

\newpage
\subsection{Ergebnisse}
Nach Abschluss der Bachelorarbeit konnte ein Grossteil der gewünschten Funktionalität implementiert werden. Das Hauptziel, welches wir uns gesetzt haben, ist die Entwicklung einer Anwendung, welche die Flipped Classroom Unterrichtsmethode ermöglicht. Aufgaben-Coach erfüllt diesen Punkt und kann so, wie es jetzt existiert verwendet werden. Wird die Idee als Startup weiter verfolgt, gibt es einige Punkte, welche verbessert werden können oder neue Features, die noch implementiert werden können.
\subsubsection{Resultate}
In den folgenden Abschnitten sind die Funktionalitäten der einzelnen Benutzergruppen aufgelistet.
\subsubsection*{Schüler}
\begin{itemize}
	\item Einzelne Lektionen anschauen
	\item Aufgaben lösen
	\item Gelöste Aufgaben verbessern
	\item Fragen im Forum stellen
\end{itemize}

\subsubsection*{Lehrer}
\begin{itemize}
	\item Übungen erfassen und bearbeiten
	\begin{itemize}
		\item Dynamisches Hinzufügen von Fragen und Hilfestellungen.
		\item Punktevergabe pro Frage.
	\end{itemize}
	\item Eigene Klassen verwalten
	\begin{itemize}
		\item Fächer einer Klasse zuweisen
		\item Aufgaben dem Wochenplan einer Klasse zuweisen
	\end{itemize}
	\item Gelöste Aufgaben korrigieren
	\begin{itemize}
		\item Filtern von allen gelösten Aufgaben seiner Klasse oder einzelner Schüler
		\item Pagination Ansicht der gefilterten Aufgaben
		\item Einzelne Aufgaben akzeptieren oder zurückweisen
		\item Feedback für einzelne Aufgaben erfassen
		\item Mail an Schüler mit dem Feedback senden
	\end{itemize}
	\item Statistiken einsehen
\end{itemize}


\subsubsection*{Admin}
Im Prinzip hat der Admin die Rechte, um alles auf der Website zu tun. Nachfolgend sind nur die Features, welche sich direkt an den Admin richten:

\begin{itemize}
	\item Schüler und Lehrer erfassen und bearbeiten
	\item Klassen erstellen und User diesen Klassen zuweisen
\end{itemize}


\subsubsection{Verbesserung}
\label{verbesserung}
Die Anwendung war viel komplexer als zu Beginn erwartet. Um im Zeitplan zu bleiben, musste bei gewissen Features  Kompromisse eingegangen werden, um im Zeitplan bleiben zu können. Folgende Punkte müssen noch verbessert werden:

\begin{itemize}
	\item Im Moment wurde beim Frontend hauptsächlich Bootstrap eingesetzt. Vereinzelt wurde auch AJAX oder jQuery eingesetzt, um die Website dynamischer zu machen. Für die Zukunft muss man evaluieren, ob man nicht besser ein anderes Frontend Framework, wie zum Beispiel React oder Angular, verwendet.
	\item Der Lehrer die Aufgaben einem Wochenplan zuweisen. Dieses Feature wurde aber nur sehr grob implementiert und kann noch verbessert werden. Der Lehrer sollte den Wochenplan für mehrere Wochen vorgeben können. Wird eine Aufgabe dem Wochenplan zugewiesen, soll eine Deadline hinterlegt werden. Nach Ablauf dieser Deadline sollen keine weiteren Abgaben gemacht werden können.
	\item Das Filtern der Aufgaben kann noch verbessert werden. Lehrer sollen auch nach Klassen oder nach archivierten Aufgaben filtern können.
	\item Im Moment sind alle Aufgaben global. Erstellt ein Lehrer eine Aufgabe, so kann diese von anderen Lehrern auch bearbeitet oder sogar gelöscht werden. In Zukunft soll ein Lehrer nur Zugriff auf seine eigenen Aufgaben haben und auch nur diese bearbeiten können.
	\item Wird der Klasse ein Fach zugewiesen, haben die Schüler automatisch Zugriff auf alle Themen, Lessons und Aufgaben. Der Lehrer soll jedoch auch einzelne Themen, Lessons oder Aufgaben freischalten können.
	\item Erstellen die Schüler einen Forumsbeitrag, erhält der Lehrer keine Benachrichtigung. Er soll jedoch eine Nachricht erhalten, so dass er auf unbeantwortete Fragen von Schülern eingehen kann.
\end{itemize}


\subsubsection{Weiterentwicklung}
In einem nächsten Schritt sollten die unter Abschnitt \ref{verbesserung} genannten Punkte verbessert werden. Es gibt jedoch viele interessante Features, welche noch zusätzlich implementiert werden können.

\begin{itemize}
	\item Ein globales Forum, in welchem die Schüler Fragen stellen können, welche nicht direkt auf eine Aufgabe bezogen sind.
	\item Quizze, welche die Schüler während des Unterrichts lösen können. So sieht der Lehrer sehr schnell, was die Schüler noch nicht verstanden haben.
	\item Das Dashboard könnte so erweitert werden, dass auch allgemeine Informationen, wie kommende Events oder Anlässe, dargestellt werden können.
	\item Intelligentes Zuweisen von Aufgaben. Im Moment können die Schüler selber sagen, an welchen Fächern oder Übungen sie arbeiten. Es könnte ein System implementiert werden, welches erkennt, in welchem Themen der Schüler Mühe hat und automatisch solche Übungen zuweisen, dass sich der Schüler verbessern kann.
	\item In einer Französisch Lektion sprechen deutschsprachige Kinder miteinander auf französisch.	Mit einer solchen Plattform wäre es möglich, dass ein deutsprachiger Lehrer eine Lektion zusammen mit einem Lehrer aus der Westschweiz plant. Während der Lektion wird jedes deutschsprachige Kind einem französisch sprechenden Kind zugewiesen, welche dann untereinander sprechen können. Dies macht vielleicht den Französisch Unterricht noch ein wenig interessanter.
\end{itemize}


\subsubsection{Nicht Funktionale Anforderungen}

\subsubsection*{Functionality}
%TODO User Interface
%TODO Security
%TODO HTTPS

\subsubsection*{Usability}
%TODO Responsiveness

\subsubsection*{Reliability}
Die von uns gestellten Anforderungen im Kapitel \ref{chapter_nfr} an die Verlässlichkeit der Applikation wurden erfüllt. Um dies sicherzustellen, wurde die Applikation auf dem von der HSR zur Verfügung gestellten Server deployed und während dreier Tage wurden Stichproben durchgeführt. Ausserdem wurden während dieser Zeit die Performance Tests durchgeführt, was beweist, dass die Applikation auch bei einer Überlastung stabil läuft.

%TODO Fautl Tolerance

\subsubsection*{Performance}
Wie im Kapitel \ref{chapter_performance_tests} und der Abbildung \ref{performance_tests} ersichtlich ist, werden die gestellten Anforderungen an die Response Time nicht ganz erfüllt. Nicht alle eingegangenen Requests konnten in weniger als einer Sekunde abgearbeitet werden.

\subsubsection*{Supportability}
%TODO Maintainability

Die Applikation kann von 300 Benutzern zeitgleich verwendet werden. Wie jedoch bei den Performance Tests zu sehen ist, kann es teilweise zu längeren Wartezeiten beim Laden einer Seite kommen.













\newpage

\newpage
\subsection{Ergebnisse}

\subsubsection{Resultate}
Trotz des verkürzten Zeitplans war es dennoch möglich, einen Grossteil der Funtkionen umzusetzen. Nachfolgend sind die Funktionen aufgelistet, welche durch Aufgaben-Coaching ermöglicht werden.

\subsubsection*{Student}
%Nachfolgend wird die Funktionalität der Schüler aufgelistet:

\begin{itemize}
	\item Einzelne Lektionen anschauen
	\item Aufgaben lösen, falls nötig auch verbessern
	\begin{itemize}
		\item Pro Frage können mehrere Hilfestellungen bereit stehen, welche dem Schüler beim Beantworten dieser Frage helfen.
	\end{itemize}
	\item Fragen im Forum stellen
\end{itemize}

\subsubsection*{Teacher}
%Lehrer haben folgende Funktionalität:

\begin{itemize}
	\item Übungen erfassen und bearbeiten
	\begin{itemize}
		\item Dynamisches hinzufügen von Fragen und Hilfestellungen.
		\item Punktevergabe pro Frage.
	\end{itemize}
	\item Eigene Klassen verwalten
	\begin{itemize}
		\item Fächer einer Klasse zuweisen
		\item Aufgaben dem Wochenplan einer Klasse zuweisen
	\end{itemize}
	\item Gelöste Aufgaben korrigieren
	\begin{itemize}
		\item Filtern von allen gelösten Aufgaben seiner Schüler
		\item Pagination Ansicht der gefilterten Aufgaben
		\item Einzelne Aufgaben akzeptieren oder zurückweisen
		\item Feedback für einzelne Aufgaben erfassen
		\item Mail an Schüler mit dem Feedback schicken
	\end{itemize}
	\item Statistiken einsehen
\end{itemize}


\subsubsection*{Admin}
Im Prinzip hat der Admin die Rechte, um alles auf der Website zu tun. Nachfolgend sind nur die Features, welche sich direkt an den Admin richten:

\begin{itemize}
	\item Schüler und Lehrer erfassen und bearbeiten
	\item Klassen erstellen und User diesen Klassen zuweisen
\end{itemize}


\subsubsection{Verbesserung}
\label{verbesserung}
Die Anwendung war viel komplexer als zu Beginn erwartet. Um im Zeitplan zu bleiben, musste bei gewissen Features  Kompromisse eingegangen werden, um im Zeitplan bleiben zu können. Folgende Punkte müssen noch verbessert werden:

\begin{itemize}
	\item Im Moment wurde beim Frontend hauptsächlich Bootstrap eingesetzt. Vereinzelt wurde auch AJAX oder jQuery eingesetzt, um die Website dynamischer zu machen. Für die Zukunft muss man schauen, ob man nicht besser ein anderes Frontend Framework verwendet.
	\item Im Moment kann der Lehrer die Aufgaben einem Wochenplan zuweisen. Dieses Feature wurde aber nur sehr grob implementiert und könnte noch verbessert werden. Der Lehrer sollte den Wochenplan für mehrere Wochen vorgeben können und die Aufgaben sollten eine Deadline bekommen und nach Ablauf nicht mehr von Schülern bearbeitet werden können.
	\item Das Filtering der Aufgaben kann noch verbessert werden. Lehrer sollen auch nach Klassen oder nach archivierten Aufgaben filtern können.
	\item Im Moment sind alle Aufgaben global. Erstellt ein Lehrer eine Aufgabe, so kann diese von anderen Lehrern auch bearbeitet oder sogar gelöscht werden. In Zukunft soll ein Lehrer nur Zugriff auf seine eigenen Aufgaben haben und auch nur diese bearbeiten können.
	\item Wir der Klasse ein Fach zugewiesen, haben die Schüler automatisch Zugriff auf alle Themen, Lessons und Aufgaben. Der Lehrer soll jedoch auch einzelne Themen, Lessons oder Aufgaben freischalten können.
	\item Erstellen die Schüler einen Forumsbeitrag, erhält der Lehrer keine Benachrichtigung. Er soll jedoch eine Nachricht bekommen, so dass er auf unbeantwortete Fragen von Schülern eingehen kann.
\end{itemize}



\subsubsection{Weiterentwicklung}
Es ist geplant, Aufgaben-Coach als Startup weiter zu verfolgen. Dazu sollen die in Abschnitt \ref{verbesserung} genannten Punkte verbessert werden. Zudem gibt es viele interessante Features, welche noch implementiert werden können.

\begin{itemize}
	\item Ein globales Forum, in welchem die Schüler Fragen stellen können, welche nicht direkt auf eine Aufgabe bezogen sind.
	\item Quizze, welche die Schüler während des Unterrichts lösen können. So sieht der Lehrer sehr schnell, was die Schüler noch nicht verstanden haben.
	\item Das Dashboard könnte so erweitert werden, dass auch allgemeine Informationen, wie kommende Events oder Anlässe, dargestellt werden können
	\item Intelligentes Zuweisen von Aufgaben. Im Moment können die Schüler selber sagen, an welchen Fächern oder Übungen sie arbeiten. Es könnte ein System implementiert werden, welches erkennt, in welchem Themen der Schüler Mühe hat und automatisch solche Übungen zuweisen, dass sich der Schüler verbessern kann.
\end{itemize}
\newpage
